
\def\slidemode{
  \documentclass[fleqn,aspectratio=169]{beamer}
\usepackage{pgfpages}
}
\def\handoutmode{
  \documentclass[handout,fleqn,aspectratio=169]{beamer}
\usepackage{pgfpages}
\pgfpagesuselayout{resize to}[a4paper,landscape,border shrink=5mm]
}


%\def\pdfmode{handoutmode}
\def\pdfmode{slidemode}

\csname\pdfmode\endcsname

\mode<presentation>
{
  \usetheme{default}
  \usecolortheme{default}
  \usefonttheme{default}
  \setbeamertemplate{navigation symbols}{}
  \setbeamertemplate{caption}[numbered]
  \setbeamertemplate{footline}[frame number]  % or "page number"
  \setbeamercolor{frametitle}{fg=white}
  \setbeamercolor{footline}{fg=black}
}





\usepackage[english]{babel}
%\usepackage[utf8x]{inputenc}
\usepackage{tikz}
\usepackage{courier}
\usepackage{array}
\usepackage{bold-extra}
%\usepackage{minted}
%\usepackage[thicklines]{cancel}
%\usepackage{fancyvrb}
\usepackage{kotex}
\usepackage{paralist}
\usepackage{collectbox}
\usepackage{amsmath}
\usepackage{mathtools}
\usepackage{nccmath}
\usepackage{bm}
\usepackage{booktabs}
\usepackage{tcolorbox}
\usepackage[absolute,overlay]{textpos}

\xdefinecolor{dianablue}{rgb}{0.18,0.24,0.31}
\xdefinecolor{darkblue}{rgb}{0.1,0.1,0.7}
\xdefinecolor{darkgreen}{rgb}{0,0.5,0}
\xdefinecolor{darkgrey}{rgb}{0.35,0.35,0.35}
\xdefinecolor{darkorange}{rgb}{0.8,0.5,0}
\xdefinecolor{darkred}{rgb}{0.7,0,0}
\definecolor{darkgreen}{rgb}{0,0.6,0}
\definecolor{mauve}{rgb}{0.58,0,0.82}



\title[]{Lecture 7: Random Processes, Part I}
\author{Yi, Yung (이융)}
\institute{EE210: Probability and Introductory Random Processes\\ KAIST EE}
\date{\today}

%%%%%%%%%%%% real, integer notation
\newcommand{\real}{{\mathbb R}}
\newcommand{\integer}{{\mathbb Z}}

%%% set, vector, matrix
\newcommand{\set}[1]{\ensuremath{\mathcal #1}}
\renewcommand{\vec}[1]{\bm{#1}}
\newcommand{\mat}[1]{\bm{#1}}


%%% big parenthesis
\def\Bl{\Bigl}
\def\Br{\Bigr}
\def\lf{\left}
\def\ri{\right}


%%% floor notations
\newcommand{\lfl}{{\lfloor}}
\newcommand{\rfl}{{\rfloor}}
\newcommand{\floor}[1]{{\lfloor #1 \rfloor}}

%%% gradient
\newcommand{\grad}[1]{\nabla #1}

%%% definition
\newcommand{\eqdef}{\ensuremath{\triangleq}}
%%% imply
\newcommand{\imp}{\Longrightarrow}



\newcommand{\separator}{
%  \begin{center}
    \par\noindent\rule{\columnwidth}{0.3mm}
%  \end{center}
}

\newcommand{\mynote}[1]{{\it \color{red} [#1]}}







%%% equation alignment
\newcommand{\aleq}[1]{\begin{align*}#1\end{align*}}

%%%%%%%%%%%%%%%% colored emphasized font, blanked words

\newcommand{\empr}[1]{{\color{red}\emph{#1}}}
\newcommand{\empb}[1]{{\color{blue}\emph{#1}}}

%normal colored text
\newcommand{\redf}[1]{{\color{red} #1}}
\newcommand{\yellowf}[1]{{\color{yellow} #1}}
\newcommand{\bluef}[1]{{\color{blue} #1}}
\newcommand{\grayf}[1]{{\color{gray} #1}}
\newcommand{\magenf}[1]{{\color{magenta} #1}}
\newcommand{\greenf}[1]{{\color{green} #1}}
\newcommand{\cyanf}[1]{{\color{cyan} #1}}
\newcommand{\orangef}[1]{{\color{orange} #1}}


\newcommand{\blk}[1]{\underline{\mbox{\hspace{#1}}}}


\newcommand{\redblk}[1]{\framebox{\color{red} #1}}
\newcommand{\redblank}[2]{\framebox{\onslide<#1->{\color{red} #2}}}
\newcommand{\blueblk}[1]{\framebox{\color{blue} #1}}
\newcommand{\blueblank}[2]{\framebox{\onslide<#1->{\color{blue} #2}}}



\makeatletter
\newcommand{\mybox}{%
    \collectbox{%
        \setlength{\fboxsep}{1pt}%
        \fbox{\BOXCONTENT}%
    }%
}
\makeatother

\makeatletter
\newcommand{\lecturemark}{%
    \collectbox{%
        \setlength{\fboxsep}{1pt}%
        \fcolorbox{red}{yellow}{\BOXCONTENT}%
    }%
}
\makeatother

\usepackage{tcolorbox}
\newcommand{\mycolorbox}[1]{
\begin{tcolorbox}[colback=red!5!white,colframe=red!75!black]
#1
\end{tcolorbox}
}

%%%% figure inclusion
\newcommand{\mypic}[2]{
\begin{center}
\includegraphics[width=#1\textwidth]{#2}
\end{center}
}

\newcommand{\myinlinepic}[2]{
\makebox[0cm][r]{\raisebox{-4ex}{\includegraphics[height=#1]{#2}}}
}


%%%% itemized and enumerated list
\newcommand{\bci}{\begin{compactitem}}
\newcommand{\eci}{\end{compactitem}}
\newcommand{\bce}{\begin{compactenum}}
\newcommand{\ece}{\end{compactenum}}


%%%% making 0.5/0.5 two columns
%%%% how to use: first number: length of separation bar
% \mytwocols{0.6}
% {
% contents in the left column
% }
% {
% contents in the right column
% }
%%%%
\newcommand{\mytwocols}[3]{
\begin{columns}[T] \column{.499\textwidth} #2 \column{.001\textwidth} \rule{.3mm}{{#1}\textheight} \column{.499\textwidth} #3 \end{columns}}

\newcommand{\mythreecols}[4]{
\begin{columns}[T] \column{.31\textwidth} #2 \column{.001\textwidth} \rule{.3mm}{{#1}\textheight} \column{.31\textwidth} #3 \column{.001\textwidth} \rule{.3mm}{{#1}\textheight} \column{.31\textwidth} #4  \end{columns}}

\newcommand{\mysmalltwocols}[3]{
\begin{columns}[T] \column{.4\textwidth} #2 \column{.001\textwidth} \rule{.3mm}{{#1}\textheight} \column{.4\textwidth} #3 \end{columns}}

%%%% making two columns with customized ratios
%%%% how to use:
%first parameter: length of separation bar
%second parameter: ratio of left column
%third parameter: ratio of right column
% \mytwocols{0.6}{0.7}{0.29}
% {
% contents in the left column
% }
% {
% contents in the right column
% }
%%%%
\newcommand{\myvartwocols}[5]{
\begin{columns}[T] \column{#2\textwidth} {#4} \column{.01\textwidth} \rule{.3mm}{{#1}\textheight} \column{#3\textwidth} {#5} \end{columns}}

%%% making my block in beamer
%%% first parameter: title of block
%%% second parameter: contents of block
\newcommand{\myblock}[2]{
\begin{block}{#1} {#2}  \end{block}}

%%% independence notation
\newcommand{\indep}{\perp \!\!\! \perp}

%%%% probability with different shapes (parenthesis or bracket) and different sizes
%%% `i' enables us to insert the subscript to the probability
\newcommand{\bprob}[1]{\mathbb{P}\Bl[ #1 \Br]}
\newcommand{\prob}[1]{\mathbb{P}[ #1 ]}
\newcommand{\cbprob}[1]{\mathbb{P}\Bl( #1 \Br)}
\newcommand{\cprob}[1]{\mathbb{P}( #1 )}
\newcommand{\probi}[2]{\mathbb{P}_{#1}[ #2 ]}
\newcommand{\bprobi}[2]{\mathbb{P}_{#1}\Bl[ #2 \Br]}
\newcommand{\cprobi}[2]{\mathbb{P}_{#1}( #2 )}
\newcommand{\cbprobi}[2]{\mathbb{P}_{#1}\Bl( #2 \Br)}

%%%% expectation with different shapes (parenthesis or bracket) and different sizes
%%% `i' enables us to insert the subscript to the expectation
\newcommand{\expect}[1]{\mathbb{E}[ #1 ]}
\newcommand{\cexpect}[1]{\mathbb{E}( #1 )}
\newcommand{\bexpect}[1]{\mathbb{E}\Bl[ #1 \Br]}
\newcommand{\cbexpect}[1]{\mathbb{E}\Bl( #1 \Br)}
\newcommand{\bbexpect}[1]{\mathbb{E}\lf[ #1 \ri]}
\newcommand{\expecti}[2]{\mathbb{E}_{#1}[ #2 ]}
\newcommand{\bexpecti}[2]{\mathbb{E}_{#1}\Bl[ #2 \Br]}
\newcommand{\bbexpecti}[2]{\mathbb{E}_{#1}\lf[ #2 \ri]}

%%%% variance
\newcommand{\var}[1]{\text{var}[ #1 ]}
\newcommand{\bvar}[1]{\text{var}\Bl[ #1 \Br]}
\newcommand{\cvar}[1]{\text{var}( #1 )}
\newcommand{\cbvar}[1]{\text{var}\Bl( #1 \Br)}

%%%% covariance
\newcommand{\cov}[1]{\text{cov}( #1 )}
\newcommand{\bcov}[1]{\text{cov}\Bl( #1 \Br)}

%%% Popular pmf, pdf notation to avoid long typing
\newcommand{\px}{\ensuremath{p_X(x)}}
\newcommand{\py}{\ensuremath{p_Y(y)}}
\newcommand{\pz}{\ensuremath{p_Z(z)}}
\newcommand{\pxA}{\ensuremath{p_{X|A}(x)}}
\newcommand{\pyA}{\ensuremath{p_{Y|A}(y)}}
\newcommand{\pzA}{\ensuremath{p_{Z|A}(z)}}
\newcommand{\pxy}{\ensuremath{p_{X,Y}(x,y)}}
\newcommand{\pxcy}{\ensuremath{p_{X|Y}(x|y)}}
\newcommand{\pycx}{\ensuremath{p_{Y|X}(y|x)}}

\newcommand{\fx}{\ensuremath{f_X(x)}}
\newcommand{\Fx}{\ensuremath{F_X(x)}}
\newcommand{\fy}{\ensuremath{f_Y(y)}}
\newcommand{\Fy}{\ensuremath{F_Y(y)}}
\newcommand{\fz}{\ensuremath{f_Z(z)}}
\newcommand{\Fz}{\ensuremath{F_Z(z)}}
\newcommand{\fxA}{\ensuremath{f_{X|A}(x)}}
\newcommand{\fyA}{\ensuremath{f_{Y|A}(y)}}
\newcommand{\fzA}{\ensuremath{f_{Z|A}(z)}}
\newcommand{\fxy}{\ensuremath{f_{X,Y}(x,y)}}
\newcommand{\Fxy}{\ensuremath{F_{X,Y}(x,y)}}
\newcommand{\fxcy}{\ensuremath{f_{X|Y}(x|y)}}
\newcommand{\fycx}{\ensuremath{f_{Y|X}(y|x)}}

\newcommand{\fth}{\ensuremath{f_\Theta(\theta)}}
\newcommand{\fxcth}{\ensuremath{f_{X|\Theta}(x|\theta)}}
\newcommand{\fthcx}{\ensuremath{f_{\Theta|X}(\theta|x)}}

\newcommand{\pkcth}{\ensuremath{p_{X|\Theta}(k|\theta)}}
\newcommand{\fthck}{\ensuremath{f_{\Theta|X}(\theta|k)}}


%%%% indicator
\newcommand{\indi}[1]{\mathbf{1}_{ #1 }}

%%%% exponential rv.
\newcommand{\elambdax}{\ensuremath{e^{-\lambda x}}}

%%%% normal  rv.
\newcommand{\stdnormal}{\ensuremath{\frac{1}{\sqrt{2\pi}} e^{-x^2/2}}}
\newcommand{\gennormal}{\ensuremath{\frac{1}{\sigma\sqrt{2\pi}} e^{-(x-\mu)^2/2}}}

%%%%%% estimator, estimate
\newcommand{\hth}{\ensuremath{\hat{\theta}}}
\newcommand{\hTH}{\ensuremath{\hat{\Theta}}}
\newcommand{\MAP}{\ensuremath{\text{MAP}}}
\newcommand{\LMS}{\ensuremath{\text{LMS}}}
\newcommand{\LLMS}{\ensuremath{\text{L}}}
\newcommand{\ML}{\ensuremath{\text{ML}}}

%%%% colored text
\newcommand{\red}[1]{\color{red}#1}
\newcommand{\cyan}[1]{\color{cyan}#1}
\newcommand{\magenta}[1]{\color{magenta}#1}
\newcommand{\blue}[1]{\color{blue}#1}
\newcommand{\green}[1]{\color{green}#1}
\newcommand{\white}[1]{\color{white}#1}

\newcommand{\defi}{{\color{red} Definition.} }
\newcommand{\proff}{{\color{red} Proof.} }
\newcommand{\exam}{{\color{red} Example.} }
\newcommand{\question}{{\color{red} Question.} }
\newcommand{\thm}{{\color{red} Theorem.} }
\newcommand{\background}{{\color{red} Background.} }
\newcommand{\msg}{{\color{red} Message.} }

\newcommand{\bern}[1]{\ensuremath{\text{Bern}(#1)} }
\newcommand{\bp}[1]{\ensuremath{\text{BP}(#1)} }
\newcommand{\poisson}[1]{\ensuremath{\text{Poisson}(#1)} }
\newcommand{\pp}[1]{\ensuremath{\text{PP}(#1)} }


%%%%%%%%%%%%%%%%%%%%%%% old macros that you can ignore %%%%%%%%%%%%%%%%%%%%%%%%

% \def\un{\underline}
% \def\ov{\overline}


% \newcommand{\beq}{\begin{eqnarray*}}
% \newcommand{\eeq}{\end{eqnarray*}}
% \newcommand{\beqn}{\begin{eqnarray}}
% \newcommand{\eeqn}{\end{eqnarray}}
% \newcommand{\bemn}{\begin{multiline}}
% \newcommand{\eemn}{\end{multiline}}
% \newcommand{\beal}{\begin{align}}
% \newcommand{\eeal}{\end{align}}
% \newcommand{\beas}{\begin{align*}}
% \newcommand{\eeas}{\end{align*}}



% \newcommand{\bd}{\begin{displaymath}}
% \newcommand{\ed}{\end{displaymath}}
% \newcommand{\bee}{\begin{equation}}
% \newcommand{\eee}{\end{equation}}


% \newcommand{\vs}{\vspace{0.2in}}
% \newcommand{\hs}{\hspace{0.5in}}
% \newcommand{\el}{\end{flushleft}}
% \newcommand{\bl}{\begin{flushleft}}
% \newcommand{\bc}{\begin{center}}
% \newcommand{\ec}{\end{center}}
% \newcommand{\remove}[1]{}

% \newtheorem{theorem}{Theorem}
% \newtheorem{corollary}{Corollary}
% \newtheorem{prop}{Proposition}
% \newtheorem{lemma}{Lemma}
% \newtheorem{defi}{Definition}
% \newtheorem{assum}{Assumption}
% \newtheorem{example}{Example}
% \newtheorem{property}{Property}
% \newtheorem{remark}{Remark}

% \newcommand{\separator}{
%   \begin{center}
%     \rule{\columnwidth}{0.3mm}
%   \end{center}
% }

% \newenvironment{separation}
% { \vspace{-0.3cm}
%   \separator
%   \vspace{-0.25cm}
% }
% {
%   \vspace{-0.5cm}
%   \separator
%   \vspace{-0.15cm}
% }

% \def\A{\mathcal A}
% \def\oA{\overline{\mathcal A}}
% \def\S{\mathcal S}
% \def\D{\mathcal D}
% \def\eff{{\rm Eff}}
% \def\bD{\bm{D}}
% \def\cU{{\cal U}}
% \def\bbs{{\mathbb{s}}}
% \def\bbS{{\mathbb{S} }}
% \def\cM{{\cal M}}
% \def\bV{{\bm{V}}}
% \def\cH{{\cal H}}
% \def\ch{{\cal h}}
% \def\cR{{\cal R}}
% \def\cV{{\cal V}}
% \def\cA{{\cal A}}
% \def\cX{{\cal X}}
% \def\cN{{\cal N}}
% \def\cJ{{\cal J}}
% \def\cK{{\cal K}}
% \def\cL{{\cal L}}
% \def\cI{{\cal I}}
% \def\cY{{\cal Y}}
% \def\cZ{{\cal Z}}
% \def\cC{{\cal C}}
% \def\cR{{\cal R}}
% \def\id{{\rm Id}}
% \def\st{{\rm st}}
% \def\cF{{\cal F}}
% \def\bz{{\bm z}}
% \def\cG{{\cal G}}
% \def\N{\mathbb{N}}
% \def\bbh{\mathbb{h}}
% \def\bbH{\mathbb{H}}
% \def\bbi{\mathbb{i}}
% \def\bbI{\mathbb{I}}
% \def\R{\mathbb{R}}
% \def\bbR{\mathbb{R}}
% \def\bbr{\mathbb{r}}
% \def\cB{{\cal B}}
% \def\cP{{\cal P}}
% \def\cS{{\cal S}}
% \def\bW{{\bm W}}
% \def\bc{{\bm c}}

% %\def\and{\quad\mbox{and}\quad}
% \def\ind{{\bf 1}}


% \def\bmg{{\bm{\gamma}}}
% \def\bmr{{\bm{\rho}}}
% \def\bmq{{\bm{q}}}
% \def\bmt{{\bm{\tau}}}
% \def\bmn{{\bm{n}}}
% \def\bmcapn{{\bm{N}}}
% \def\bmrho{{\bm{\rho}}}

% \def\igam{\underline{\gamma}(\lambda)}
% \def\sgam{\overline{\gamma}(\lambda)}
% \def\ovt{\overline{\theta}}
% \def\ovT{\overline{\Theta}}
% \def\PP{{\mathrm P}}
% \def\EE{{\mathrm E}}
% \def\iskip{{\vskip -0.4cm}}
% \def\siskip{{\vskip -0.2cm}}

% \def\bp{\noindent{\it Proof.}\ }
% \def\ep{\hfill $\Box$}



\begin{document}

%itemshape
\setbeamertemplate{itemize item}{\scriptsize\raise1.25pt\hbox{\donotcoloroutermaths$\bullet$}}
\setbeamertemplate{itemize subitem}{\tiny\raise1.5pt\hbox{\donotcoloroutermaths$\circ$}}
\setbeamertemplate{itemize subsubitem}{\tiny\raise1.5pt\hbox{\donotcoloroutermaths$\blacktriangleright$}}
%default value for spacing
\plitemsep 0.1in
\pltopsep 0.03in
\setlength{\parskip}{0.15in}
%\setlength{\parindent}{-0.5in}
\setlength{\abovedisplayskip}{0.07in}
\setlength{\mathindent}{0cm}
\setbeamertemplate{frametitle continuation}{[\insertcontinuationcount]}

\setlength{\leftmargini}{0.5cm}
\setlength{\leftmarginii}{0.5cm}

\setlength{\fboxrule}{0.05pt}
\setlength{\fboxsep}{5pt}

\begin{frame}
  \titlepage
\end{frame}

% % Uncomment these lines for an automatically generated outline.
% \begin{frame}{Outline}
% % \tableofcontents
% \plitemsep 0.1in
% \bci
% \item

% \item
% \eci
% \end{frame}

% START START START START START START START START START START START START START

%%%%%%%%%%%%%%%%%%%%%%%%%%%%%%%%%%%%%%%%%%%%%%%%%%%%%%
\begin{frame}{Roadmap}

\plitemsep 0.2in

\bce[(1)]
% \item[$\circ$] A lot of applications in engineering systems


% \bigskip

\item Introduction of Random Processes
\item Bernoulli Processes
\item Poisson Processes: Poisson RV and Basic Idea

\item Poisson Processes: Definition and Properties

\item Playing with Bernoulli and Poisson Processes

\item Random Incidence Phenomenon

\ece

\end{frame}

%%%%%%%%%%%%%%%%%%%%%%%%%%%%%%%%%%%%%%%%%%%%%%%%%%%%%%
\section{L8(1)}
\begin{frame}{Roadmap}

\plitemsep 0.2in

\bce[(1)]

\item \redf{Introduction of Random Processes}
\item \grayf{Bernoulli Processes
\item Poisson Processes: Poisson RV and Basic Idea

\item Poisson Processes: Definition and Properties

\item Playing with Bernoulli and Poisson Processes

\item Random Incidence Phenomenon

}

\ece

\end{frame}


%%%%%%%%%%%%%%%%%%%%%%%%%%%%%%%%%%%%%%%%%%%%%%%%%%%%%%
\begin{frame}{Things that evolve in time}


\mytwocols{0.75}
{
\plitemsep 0.1in
\bci
\item<1-> Many probabilistic experiments that \redf{evolve in time}
\bci
\item<2-> Sequence of daily prices of a stock
\item<2-> Sequence of scores in football
\item<2-> Sequence of failure times of a machine
\item<2-> Sequence of traffic loads in Internet
\eci

\item<3-> Random process is a mathematical model for it.

%\item Each numerical value in the sequence is a random variable.


\eci
}
{
  \vspace{-0.5cm}
  \scriptsize
  \centering
  \mypic{0.6}{L8_bitcoin.png}
  \centering (a) Prices of a crytocurrency
  \mypic{0.6}{L8_nettraffic.png}
  \centering (b) Internet traffic traces
}
\end{frame}


%%%%%%%%%%%%%%%%%%%%%%%%%%%%%%%%%%%%%%%%%%%%%%%%%%%%%%
\begin{frame}{Random Process: Basics (1)}

\plitemsep 0.1in
\bci

\item<2-> A random process is a \redf{sequence} of random variables indexed by \redf{time.}

\item<3-> Time: discrete or continuous (a modeling choice in most cases)

\item<4-> Notation

- $(X_t)_{t \in \set{T}}$ or $\Bl(X(t)\Bl)_{t \in \set{T}},$ where $\set{T} = \real$ (continuous) or $\set{T} = \{ 0, 1, 2, \ldots \}$ (discrete)

\onslide<5->{- For the discrete case, we also often use $(X_n)_{n \in \integer_+}.$}

\onslide<6->{- We will use all of them, unless confusion arises. }

\item<7-> For a \magenf{fixed} time $t$, $X_t$ (or $X(t)$) is a random variable.

\item<8-> The values that $X_t$ (or $X(t)$) can take: discrete or continuous

\eci

\end{frame}



%%%%%%%%%%%%%%%%%%%%%%%%%%%%%%%%%%%%%%%%%%%%%%%%%%%%%%
\begin{frame}{Random Processes: Basics (2)}
  
\plitemsep 0.05in
\bci
  
\item<1-> \exam Discrete time RP. $\{ X_1, X_2, X_3, \ldots, \}.$ 
  
\bci
\item<2-> $X_i$: \# of Covid-19 infections at day  $i$ in South Korea, which is a random variable.
\item<3-> Then, $X_i: \Omega\mapsto \real,$ where the sample space
  $\Omega$ is the set of all outcomes. 
  
\item<4-> An outcome $\omega \in \Omega$ is a infinite sequence of infections


\item<5-> For example, $\omega_1 = (100,150, 130, \ldots)$, $\omega_2 = (200,300, 400, \ldots)$
\item<6-> $X_3(\omega_1) = 130,$ $X_2(\omega_2) = 300,$ $X_1(\omega_2) =  200,$ etc.
  
  \eci

  \item<7-> \exam Continuous time RP. $(X(t))_{t \in \real^+}$ 
  
\bci
\item<8-> $X(t)$: bitcoin price at time $t$, which is a random variable.
\item<9-> Then, $X(t): \Omega\mapsto \real.$ 
  
\item<10-> An outcome $\omega \in \Omega$ is a trajectory of prices over $[0,\infty)$

\item<11-> $X(3.7, \omega_1) = 3409,$ $X(2,\omega_2) = 5000,$ $X(7.8,\omega_3) = 2800,$ etc.
  
  \eci



  \eci
\end{frame}


%%%%%%%%%%%%%%%%%%%%%%%%%%%%%%%%%%%%%%%%%%%%%%%%%%%%%%
\begin{frame}{Random Processes: Our Interest}
  
\plitemsep 0.07in
\bci
  
 \item<2-> \redf{Question.} Already studied a sequence (or a
   collection) of rvs $X_1, X_2, \ldots, X_n.$ What's the difference?

-  Assume a discrete time random process for our discussion.
  
\item<3-> \bluef{Physical difference:} \redf{infinite} sequence of $X_1, X_2, \ldots, .$
  \bci
\item<4-> Sample space? set of all outcomes?
\item<5-> an outcome: an infinite sequence of sample values $x_1, x_2, \ldots,$ 
\eci

\item<6-> \bluef{Semantic difference:} Understand $i$ in $X_i$ as
  time. Also, interesting questions are asked from the random process point of view.

  \bci
\item<7-> \magenf{Dependence}: How $X_1, X_2, \ldots$ are related to each other
  as a time series. Prediction of values in the future.
\item<8-> \magenf{Long-term behavior}: What is the fraction of times that a
  stock price is above 3000?
\item<9-> Other interesting questions, depending on the target random process
  \eci
  
\eci

\end{frame}


%%%%%%%%%%%%%%%%%%%%%%%%%%%%%%%%%%%%%%%%%%%%%%%%%%%%%%

\begin{frame}{4 Types of Random Processes}

- Types of time and value

\bigskip
\mytwocols{0.5}
{
\bigskip
\small
\plitemsep 0.1in
\bce[(a)]
\item continuous time, continuous value
\item continuous time, discrete value
\item discrete time, continuous value
\item discrete time, discrete value
\ece

}
{
\centering
\mypic{0.95}{L8_RP_types.jpg}

}

\end{frame}



%%%%%%%%%%%%%%%%%%%%%%%%%%%%%%%%%%%%%%%%%%%%%%%%%%%%%%



\begin{frame}{Random Processes in This Course}

  \mythreecols{0.7}
  {
    \scriptsize
    \plitemsep 0.03in
    
    \bci
  \item<1-> The simplest RP
  \item<1-> discrete time 
  \item<2-> $X_i \indep \{X_{i-1}, X_{i-2}, \ldots, X_1 \}$
  \item<3-> \magenf{Bernoulli Process (BP)}
  \item<10-> \redf{``today'' independent of ``past''}
    \eci

    \vspace{1.2cm}
    \centering
    Jacob Bernoulli (1654 - 1705), Swiss 
    \mypic{0.6}{L8_bernoulli2.jpg}
    % Daniel Bernoulli (1700 - 1782), Swiss 
    % \mypic{0.7}{L8_bernoulli.jpg}
  }
  {
    \scriptsize
    \plitemsep 0.03in
    
    \bci
  \item<4-> The simplest RP
  \item<4-> Continous time version of BP
  \item<5-> $[X(s)]_{s=0}^{t} \indep [X(s)]_{s=t}^{t+a}$
  \item<6-> \magenf{Poisson Process (PP)}

\item<10-> \redf{``today'' independent of ``past''}
  \eci

  \vspace{1.2cm}
    \centering
    Simeon Denis Poisson (1781 - 1840), France
    \mypic{0.5}{L8_poisson.jpg}
  }
  {
    \scriptsize
    \plitemsep 0.03in
    
    \bci
  \item<7-> One-step more general than BP/PP
  \item<7-> discrete time
  \item<8-> $X_i$ depends on $X_{i-1}$, but $\indep \{X_{i-2}, X_{i-2}, \ldots, X_1 \}$
  \item<9-> \magenf{Markov Chain (MC)}
    \item<10-> \redf{``today'' depends only on ``yesterday''}
    \eci

  \vspace{0.2cm}
    \centering
    Andrey Markov (1856 - 1922), Russia
    \mypic{0.5}{L8_markov.jpg}
  }



\end{frame}


%%%%%%%%%%%%%%%%%%%%%%%%%%%%%%%%%%%%%%%%%%%%%%%%%%%%%%
\section{L8(2)}
\begin{frame}{Roadmap}

\plitemsep 0.2in

\bce[(1)]

\item \grayf{Introduction of Random Processes}
\item \redf{Bernoulli Processes}

  \item \grayf{Poisson Processes: Poisson RV and Basic Idea

\item Poisson Processes: Definition and Properties

\item Playing with Bernoulli and Poisson Processes

\item Random Incidence Phenomenon

}

\ece

\end{frame}

%%%%%%%%%%%%%%%%%%%%%%%%%%%%%%%%%%%%%%%%%%%%%%%%%%%%%%
\begin{frame}{Bernoulli Process}

\plitemsep 0.05in
\bci

\item<1-> At each ``minute'', we toss a coin with probability of head $0<p<1.$

\bci
\item<2-> Sequence of lottery wins/looses
\item<2-> Customers (each second) to a bank
\item<2-> Clicks (at each time slot) to server
\eci


\item<3-> A sequence of \redf{independent Bernoulli trials} $X_1, X_2, \ldots, $

- We call index 1, 2, \ldots \redf{time slots} (or simply slots)

\mypic{0.5}{L8_bernoulli_ex.png}

\item<4-> Discrete time, discrete value

\item<5-> One of the simplest random processes

\item<5-> A type of ``arrival" process

%\item<6-> Next: Key questions and answers about Bernoulli process
  
\eci

\end{frame}

%%%%%%%%%%%%%%%%%%%%%%%%%%%%%%%%%%%%%%%%%%%%%%%%%%%%%%
\begin{frame}{Bernoulli Process: Questions}

\vspace{1cm}

{\Large Please pause the video and write down the questions that you
  want to ask about Bernoulli process. \hfill \lecturemark{VIDEO PAUSE}}

\bce[\bf Q1.]
\item
\item
\item
\item
\item
\ece

\end{frame}

\begin{comment}
%%%%%%%%%%%%%%%%%%%%%%%%%%%%%%%%%%%%%%%%%%%%%%%%%%%%%% 
\begin{frame}{Bernoulli Process as a Random Process}

\plitemsep 0.03in
\bci

\item<1-> \redf{Question.} We've already studied a sequence of Bernoulli rvs $X_1, X_2, \ldots, X_n.$ What's the difference?

\item<2-> \bluef{Physical difference:} \redf{infinite} sequence of $X_1, X_2, \ldots, .$
\bci
\item<3-> Sample space? set of all outcomes?
\item<4-> an outcome: an infinite sequence of sample values $x_1, x_2, \ldots,$ e.g., (0,1,1,0,0,1, \ldots)
\eci

\item<5-> \bluef{Semantic difference:} Understand $i$ in $X_i$ as time. Also, interesting questions from the random process point of view.
\bci
\item<6-> Dependence: How $X_1, X_2, \ldots$ are related to each other as a time series
\item<7-> Long-term behavior: What is the fraction of times that a machine is idle?
\item<8-> Other interesting questions, depending on the target random process
\eci

\item<9-> Next: Key questions and answers about Bernoulli process
\eci

\end{frame}

\end{comment}


%%%%%%%%%%%%%%%%%%%%%%%%%%%%%%%%%%%%%%%%%%%%%%%%%%%%%%

\begin{frame}{Number of arrivals and Time until the first arrival}

\mytwocols{0.6}
{
\plitemsep 0.1in

\onslide<1->{\redf{(Q1)} \# of arrivals in the first $n$ slots?}
\bci
\item<2-> $S_n = X_1 + X_2 +\cdots + X_n$

\item<3-> $S_n \sim \text{Binomial}(n,p)$

\item<4-> $\cexpect{S_n} = np$, $\cvar{S_n} = np(1-p)$

\item<5-> This will hold for any $n$ consecutive slots.


  \eci

\bigskip
\onslide<5->{\redf{(Q2)} \# of slots $T_1$ until the first arrival?}
\bci
\item<6-> $T_1 \sim \text{Geom}(p)$
\item<7-> $\cexpect{T_1} = 1/p,$ $\cvar{T_1} = \frac{1-p}{p^2}$
\eci
}
{
\plitemsep 0.1in
\bci
\item<8-> $T_1$ is geometric? \bluef{Memoryless}

\item<9-> Conditioned no first arrival at some time, what's the distribution of the remaining time until the first arrival?

\item<9-> Still, geometric.

\item<10-> But, more than that, as we will see. Independence across time slots leads to many useful properties, allowing the quick solution of many problems.

\eci

}

\end{frame}

%%%%%%%%%%%%%%%%%%%%%%%%%%%%%%%%%%%%%%%%%%%%%%%%%%%%%%
\begin{frame}{Memoryless and Fresh-start after \yellowf{Deterministic} time $n$}

Independence across slots $\implies$ the fresh-start anytime
when I look at the process?


  \plitemsep 0.05in

\onslide<1->{\redf{(Q3)} $U=X_1 + X_2 \indep V=X_5+X_6$?}
\bci
\item<2-> Yes

\item<2-> Because $X_i$s are independent

\eci

\medskip
\onslide<3->{\redf{(Q4)} After time $n=6,$ I start to look at the process $(X_n)_{n=6}^\infty$?}
\bci
\item<4-> $(X_1, \ldots, X_5) \indep$  $(X_n)_{n=6}^\infty$
\item<5-> \redf{Fresh-start} after a deterministic time $n$ (doesn't
  matter what happened until $n=5$.
  
\item<6-> If you watch the on-going Bernoulli process$(p)$ from some time $n,$ you still see the same Bernoulli process$(p).$
\eci

\end{frame}
%%%%%%%%%%%%%%%%%%%%%%%%%%%%%%%%%%%%%%%%%%%%%%%%%%%%%%
\begin{frame}{Fresh-start after Random time $N$ (1)}

\onslide<1->{\redf{(Q5)} The process $(X_{N},X_{N+1}, X_{N+2}, \ldots)$? Fresh-start even after \bluef{random} $N$?}

\vspace{-0.3cm}
\plitemsep 0.03in
\bci
\item<1-> This means that the time I start to look at the process is a
  random variable.

\item<2-> Examples of $N$
\mytwocols{0.25}
{
\small
\bce[\bf E1.]
\item<3-> Time of 3rd arrival
\item<4-> First time when 3 consecutive arrivals have been observed
\item<5-> Time just before 3 consecutive arrivals
\ece
}
{
\vspace{-0.7cm}
\centering
\mypic{0.7}{L8_randomtime.png}
}

\item<6-> Difference of $N$ from $n$

- The time when I watch the on-going Bernoulli process is \redf{random.}

- $N$ is a random variable, i.e., $N: \Omega \mapsto \real.$
  What is $\Omega$?

\item<7-> Do we experience the fresh-start for any $N$? {\bf E1, E2, and E3?}
\eci

\end{frame}

%%%%%%%%%%%%%%%%%%%%%%%%%%%%%%%%%%%%%%%%%%%%%%%%%%%%%%
\begin{frame}{Fresh-start after Random $N$ (2)}


\redf{(Q5)} The process $(X_{N},X_{N+1}, X_{N+2}, \ldots)$? Fresh-start even after random $N$?

\medskip
\mysmalltwocols{0.27}
{
\small
\bce[\bf E1.]
\item Time of 3rd arrival
\item First time when 3 consecutive arrivals have been observed
\item Time just before 3 consecutive arrivals
\ece
}
{
\vspace{-0.5cm}
\centering
\mypic{0.8}{L8_randomtime.png}
}


\plitemsep 0.05in
\bci

\item<2->[\bf E1.] When I watch the process, $N$ has been already determined. \redf{Yes}
\item<3->[\bf E2.] Same as {\bf E1.} \redf{Yes}
\item<4->[\bf E3.] Need the future knowledge. `111' does not become random.  \redf{No}

\medskip
\item<5-> The question of $N=n$? can be answered just from the knowledge about $X_1, X_2, \ldots, X_n$? Then, Yes! (see pp. 301 for more formal description)

\eci
\end{frame}

%%%%%%%%%%%%%%%%%%%%%%%%%%%%%%%%%%%%%%%%%%%%%%%%%%%%%%
\begin{frame}{Stopping Time}


\plitemsep 0.1in
\bci

\item<2-> In probability theory, a random time $N$ is said to be a
  \redf{stopping time}, if the question of ``$N=n$?'' can be answered
  only from the present and the past knowledge of $X_1, X_2, \ldots, X_n.$ 

\item<3-> \url{https://en.wikipedia.org/wiki/Stopping_time}

\item<4-> Fresh-start after $N$ in Bernoulli process? \magenf{Yes, if $N$ is a stopping time.}

\item<5-> Please think about two examples of stopping time and not. \hfill
  \lecturemark{VIDEO PAUSE}
  \bci
  
\item<6-> \redf{Yes.} Time when 10 consecutive arrivals have been observed
\item<7-> \redf{No.} Time of 2nd arrival in 10 consecutive arrivals
  \eci
  
  \eci
\end{frame}

%%%%%%%%%%%%%%%%%%%%%%%%%%%%%%%%%%%%%%%%%%%%%%%%%%%%%%
\begin{frame}{Distribution of Busy Periods (1)}

\plitemsep 0.05in
\bci
\item Regard an arrival as a server being busy (just for our easy understanding)

\item<2-> \redf{First busy period $B_1$}: starts with the first busy slot and ends just before the first subsequent idle slot

\onslide<2->{
\begin{center}
\mypic{0.4}{L8_busy_idle.png}
\end{center}
}

\item<3-> \redf{(Q6)} $B_1$ is a random variable. Distribution of $B_1$?


\item<4-> $N$: time of the first busy slot. $N$ is a stopping time? 

  \bci
  \item<5-> \magenf{Yes}. Thus, \bluef{fresh-start after $N.$} Because we can answer the question of $N=n$?, just using
    $X_1, X_2, \ldots, X_n$.
  \eci
\item<6-> $B_1$ is geometric with parameter \orangef{$(1-p)$}

\eci
\end{frame}

%%%%%%%%%%%%%%%%%%%%%%%%%%%%%%%%%%%%%%%%%%%%%%%%%%%%%%
\begin{frame}{Distribution of Busy Periods (2)}

  \mypic{0.4}{L8_busy_idle.png}
  
  \plitemsep 0.1in
  \bci
\item \question What about the second busy period $B_2$? 
  
\item<2-> $N$: time of the first busy slot of the second busy
  period. $N$ is a stopping time? 


\item<3-> \magenf{Yes}. Thus, fresh-start after $N.$ 

\item<4-> Then, $B_1$ and $B_2$ are \bluef{identically distributed} as $\text{Geom}(1-p).$
\item<5-> $B_3, B_4, \ldots$? 
    \eci
\end{frame}

%%%%%%%%%%%%%%%%%%%%%%%%%%%%%%%%%%%%%%%%%%%%%%%%%%%%%% 
\begin{frame}{Time of $k$-th arrival}

\plitemsep 0.1in
\bci
\item Time of the first arrival $Y_1 \sim \text{Geom}(p)$

\item<2->[] \redf{(Q7)} Time of the $k$-th arrival $Y_k$? \hfill
  \lecturemark{VIDEO PAUSE}

\medskip
\myvartwocols{0.25}{0.6}{0.37}
{
\bigskip
\small
\onslide<3->{- $T_k = Y_k - Y_{k-1}$: $k$-th inter-arrival ($k\ge 2,$ $T_1 = Y_1$)}

\onslide<4->{- $Y_k = T_1 + T_2 + \ldots + T_k.$}
}
{
\centering
\onslide<3->{\mypic{0.8}{L8_interarrival.png}}
}

\item<5-> After each $T_k,$ the fresh-start occurs.

\item<6-> $\{T_i\}$ are i.i.d. and $\sim \text{Geom}(p)$

\item<7-> We know $Y_k$'s expectation and variance: $\expect{Y_k} =
  \frac{k}{p},$ $\var{Y_k} = \frac{k(1-p)}{p^2}$, but \magenf{its distribution}?
\eci
\end{frame}



%%%%%%%%%%%%%%%%%%%%%%%%%%%%%%%%%%%%%%%%%%%%%%%%%%%%%%
\begin{frame}{PMF of $Y_k$}

\plitemsep 0.1in
\bci
\item<1-> $Y_k = T_1 + T_2 + \ldots + T_k.$

\item<1-> $\{T_i\}$ are i.i.d. and $\sim \text{Geom}(p)$
\aleq{
\onslide<2->{\cprob{Y_k = t} &= \cbprob{X_k = 1 \text{ and $k-1$ arrivals during the first $t-1$ slots} }}\cr
\onslide<3->{&=\cbprob{X_k = 1} \cdot \cbprob{\text{$k-1$ arrivals during the first $t-1$ slots}}}\cr
\onslide<4->{&= p \times {t-1 \choose k-1} p^{k-1} (1-p)^{t-k} = {t-1 \choose k-1} p^{k} (1-p)^{t-k}, \quad {t = k, k+1, \ldots}}
}

% \item<5-> $Y_k$ is called \redf{Pascal rv} with parameter $(k,p).$

% \item<6-> $\text{Pascal}(1,p)$ = $\text{Geom}(p)$
\eci
\end{frame}

%%%%%%%%%%%%%%%%%%%%%%%%%%%%%%%%%%%%%%%%%%%%%%%%%%%%%%
\begin{frame}{Pascal Random Variable with Parameter $(k,p)$}

  \plitemsep 0.2in
  \bci
\item In the sequence of Bernoulli trials, the time $Y_k$ of $k$-th success
  
\item<2-> PMF of $Y_k$
  \mycolorbox{
\centering
$$   \cprob{Y_k = t} =
\begin{cases}
{t-1 \choose k-1} p^{k} (1-p)^{t-k}, & \text{if} \quad {t = k, k+1,  \ldots}\cr
0, & \text{if} \quad {t = 1, 2, \ldots k-1} 
\end{cases}
    $$
  }
 \item<3-> $\text{Pascal}(1,p)$ = $\text{Geom}(p)$

   \eci

 \end{frame}

 %%%%%%%%%%%%%%%%%%%%%%%%%%%%%%%%%%%%%%%%%%%%%%%%%%%%%%
\section{L8(3)}
\begin{frame}{Roadmap}

\plitemsep 0.2in

\bce[(1)]

\item \grayf{Introduction of Random Processes}
\item \grayf{Bernoulli Processes}

\item \redf{Poisson Processes: Poisson RV and Basic Idea}

\item \grayf{Poisson Processes: Definition and Properties}

\item \grayf{Playing with Bernoulli and Poisson Processes}

\ece

\end{frame}



%%%%%%%%%%%%%%%%%%%%%%%%%%%%%%%%%%%%%%%%%%%%%%%%%%%%%%
\begin{frame}{Background: Poisson rv $X$ with parameter $\lambda$ (1)}

%\setdefaultleftmargin{0.1em}{}{}{}{}{}.

\plitemsep 0.05in
\bci 
\item<1-> A random variable $S \sim \text{Bin}(n,p)$: Models the number of successes in a given number $n$ of independent trials with success probability $p.$
$$
    p_S(k)=\frac{n!}{(n-k)! k!} p^k(1-p)^{n-k}
$$


\item<2-> Our interest: very large $n$ and very small
  $p,$ such that $np =\lambda,$ i.e.,  $\displaystyle \lim_{n \rightarrow \infty} p_S(k)$?

  \aleq
  {
    p_S(k) &= \onslide<3->{\frac{n(n-1)\cdots (n-k+1)}{k!}\cdot \frac{\lambda^k}{n^k}\cdot
    \left(1- \frac{\lambda}{n}\right)^{n-k}} \cr
    & \onslide<4->{= \frac{n}{n}\cdot
    \frac{(n-1)}{n}\cdots \frac{(n-k+1)}{n} \cdot \frac{\lambda^k}{k!}
    \cdot \left(1- \frac{\lambda}{n}\right)^{n} \cdot \left(1-
      \frac{\lambda}{n}\right)^{-k}} \onslide<5->{\xrightarrow{n \rightarrow \infty} e^{-\lambda}\frac{\lambda^k}{k!}}
  }


\eci 

\end{frame}

%%%%%%%%%%%%%%%%%%%%%%%%%%%%%%%%%%%%%%%%%%%%%%%%%%%%%%
\begin{frame}{Background: Poisson rv $X$ with parameter $\lambda$ (2)}

\plitemsep 0.2in
\bci 
\item<2-> A Poisson random variable $Z$ with parameter $\lambda$ takes
  nonnegative integer values, whose PMF is:
  $$
  p_Z(k) =e^{-\lambda}\frac{\lambda^k}{k!}, \quad k=0,1, \ldots
  $$
  
% \item<4-> Is this a legitimate PMF?
% \aleq{
% \sum_{k=0}^\infty e^{-\lambda}\frac{\lambda^k}{k!} = e^{-\lambda} \left(1+ \lambda + \frac{\lambda^2}{2!} + \frac{\lambda^3}{3!} \ldots \right) = e^{-\lambda} e^\lambda = 1
% }

% \item<5-> Prove this:
% $$
% \lim_{n \rightarrow \infty} p_X(k) = {n \choose k} (1/n)^k (1-1/n)^{n-k} = e^{-\lambda}\frac{\lambda^k}{k!}
% $$

\item<3-> Infinitely many slots ($n$) with the infinitely small slot
  duration (thus infinitely small success probabilty $p = \lambda/n$)

\item<4-> $\cexpect{Z} = \lambda$ (because $\lambda = np$ is the mean of
  binomial rv)

\item<5-> $\cvar{Z} = \lambda$ (because $np(1-p)$ is the variance of
  binomial rv)

  \eci 

\end{frame}

%%%%%%%%%%%%%%%%%%%%%%%%%%%%%%%%%%%%%%%%%%%%%%%%%%%%%%
\begin{frame}{Example: Poisson Approximation}
  
  \plitemsep 0.1in
  \bci 
\item A packet consisting of a string of $n$ symbols is transmitted
  over a noisy channel.
  
\item<2-> Each symbol: errorneous transimission with
  probability of 0.0001, independent of other symbols. Incorrect
  transmission is when at least one symbol is in error. 
  
\item<3-> \question How small should $n$ be in order for the probability of
  incorrect transmission to be less than 0.001?
  
\item<4-> $p$ is very small, and $n$ is reasonably large $\rightarrow$
  Poisson approximation
  
\item<5-> Prob. of incorrect transmission = $1 - \cprob{\text{no symbol
      error}} = 1-e^{-\lambda}= 1-e^{-0.0001n} < 0.0001$  

\item<6-> $n < \frac{-\ln 0.999}{0.0001} = 10.005$
  \eci 
  
\end{frame}



%%%%%%%%%%%%%%%%%%%%%%%%%%%%%%%%%%%%%%%%%%%%%%%%%%%%%%
\begin{frame}{Design of Continuous Analogue of Bernoulli Process}

\plitemsep 0.07in
\bci
\item \redf{Remind.} Geometric vs. Exponential \hfill \lecturemark{L4(3)}
  \bci
\item Two rvs with memoryless property
\item continuous system = discrete system with infintely many slots
  whose duration is infinitley small. 
  \eci
  
%  \item<1-> Very useful to both continuous and discrete random processes that are ``twins" and share the key properties.

\item<2-> Independence between what happens in a different time region

\item<3-> Memoryless and fresh-start property

\item<4-> Choose one of discrete or continuous versions for our modeling convenience.

\item<5-> \redf{Question.} How do we design the continuous analog of Bernoulli process?

\onslide<6->{- Key idea: Making it as a \redblank{7}{limit} of a sequence of Bernoulli processes}

\item<8-> Need a ``modeling sense" to make this possible. It's a good
  practice for engineers! 
\item<8-> \lecturemark{VIDEO PAUSE} 
  \eci
\end{frame}

%%%%%%%%%%%%%%%%%%%%%%%%%%%%%%%%%%%%%%%%%%%%%%%%%%%%%%
\begin{frame}{Key Design Idea to Develop a Continuous Twin (1)}

\plitemsep 0.1in
\bci
\item<1-> Continuous twin

\bci
\item<2-> Key point: Understand the number of arrivals over a given interval $[0,\tau].$

\item<3-> Assume that it has some arrival rate $\lambda$ (\# of arrivals/unit time).

\item<4-> We know how to handle Bernoulli process with discrete time slots.
\eci


\item<5-> Divide $[0,\tau]$ into slots whose length $=\delta.$ Then, $n= \# \text{ of slots} = \frac{\tau}{\delta}.$
\begin{center}
\mypic{0.4}{L8_tau_delta.jpg}
\end{center}

\item<6-> What's the limit as $\delta \rightarrow 0$ (equivalently, $n \rightarrow \infty$)

\eci
\end{frame}

%%%%%%%%%%%%%%%%%%%%%%%%%%%%%%%%%%%%%%%%%%%%%%%%%%%%%%
\begin{frame}{Key Design Idea to Develop a Continuous Twin (2)}

\begin{center}
\mypic{0.4}{L8_tau_delta.jpg}
\end{center}
\vspace{-0.8cm}
\plitemsep 0.1in
\bci
\item<1-> Now, our design idea: during one time slot of length $\delta$,
\mytwocols{0.3}
{
\aleq{
\onslide<2->{\cprob{\text{1 arrival}} &\propto \text{arrival rate and slot length}} \cr
\onslide<3->{\cprob{\text{$\ge$2 arrivals}} & \propto \text{something, but very small }}\cr
& \onslide<4->{\quad \text{   for small sloth length}\cr
\cprob{\text{0 arrival}} &= 1- \cprob{\text{1 arrival or $\ge$2 arrivals} }} \cr
}
}
{
\aleq{
\onslide<5->{\cprob{\text{1 arrival}} &= \lambda \delta}  \onslide<8->{+\magenf{o(\delta)}}\cr
\onslide<6->{\cprob{\text{$\ge$2 arrivals}} &=} \onslide<6-7>{0} \onslide<8->{+\magenf{o(\delta)}}\cr
\onslide<7->{\cprob{\text{0 arrival}} &= 1-  \lambda \delta} \onslide<8->{+ \magenf{o(\delta)}}
}
}
\vspace{-0.3cm}
\item<9-> $o(\delta)$: some function that \bluef{goes to zero faster than $\delta$}.

- Thus, for very small $\delta,$ $o(\delta)$ becomes \bluef{negligible},
compared to $\delta$.

- Example: $o(\delta)= \delta^\alpha,$ where any $\alpha >1$



\eci
\end{frame}

%%%%%%%%%%%%%%%%%%%%%%%%%%%%%%%%%%%%%%%%%%%%%%%%%%%%%%
\begin{frame}{Key Design Idea to Develop a Continuous Twin (3)}

\begin{center}
\mypic{0.4}{L8_tau_delta.jpg}
\end{center}
\vspace{-0.8cm}
\plitemsep 0.1in
\bci
\item<1-> Our interest: probability of $k$ arrivals over $[0,\tau]$

\item<2-> Given ``small" $\delta,$ \# of arrivals $\sim \text{Bin}(n,p),$ where $n = \tau/\delta$ and $p =\lambda \delta$

\item<3-> As $\delta \rightarrow 0$, $np = \tau/\delta \times \lambda \delta = \lambda \tau.$

\item<4-> \magenf{\# of arrivals over $[0,\tau]$, $\sim \text{Poisson}(\lambda \tau)$}

\item<5-> This is a continuous twin process of Bernous process, which we call \redf{Poisson process.}

\eci
\end{frame}


 %%%%%%%%%%%%%%%%%%%%%%%%%%%%%%%%%%%%%%%%%%%%%%%%%%%%%%
\section{L8(4)}
\begin{frame}{Roadmap}

\plitemsep 0.2in

\bce[(1)]

\item \grayf{Introduction of Random Processes}
\item \grayf{Bernoulli Processes}

\item \grayf{Poisson Processes: Poisson RV and Basic Idea}

\item \redf{Poisson Processes: Definition and Properties}

\item \grayf{Playing with Bernoulli and Poisson Processes}

\item \grayf{Random Incidence Phenomenon}
  
\ece

\end{frame}


%%%%%%%%%%%%%%%%%%%%%%%%%%%%%%%%%%%%%%%%%%%%%%%%%%%%%%
\begin{frame}{Poisson Process: Definition (1)}

An arrival process is called a \bluef{Poisson process} with rate
$\lambda$, if the following are satisfied:
\mycolorbox{
\small
  \plitemsep 0.05in
\bci
\item<2-> \redf{(Independence)}
  Let $N_\tau$ be the number of arrivals over the interval $[0,\tau].$ For
  any $\tau_1, \tau_2 >0,$ $N_{s+\tau_1} - N_s$ is independent of
  $N_{t+\tau_2} - N_t,$ if $t > s+ \tau_1.$

\medskip

  $\circ$ \magenf{The number of arrivals over two disjoint intervals are independent.}

\item<3-> \redf{(Time homogeneity)} For any $s,$ the distribution of
  $N_{s+\tau} - N_s$ is equal to that of $N_\tau.$ 

\medskip

  $\circ$  \magenf{$N_\tau$ becomes the number of arrivals over \bluef{any} interval of
  length $\tau.$}

% \onslide<3->{- Thus, $N_s$ can be a random variable over \bluef{any} interval of length $s.$}

\item<4-> \redf{(Small interval probability)} Let $\cprob{k,\tau} =
  \cprob{N_\tau = k},$ which satisfy:
\vspace{-0.2cm}
  \aleq{
\cprob{0,\tau} &= 1-  \lambda \tau + o(\tau) \cr
\cprob{1, \tau} &= \lambda \tau + o_1(\tau)\cr
\cprob{k,\tau} &=  o_k(\tau)\quad \text{for $k=2, 3, \ldots$}, \quad
\text{where} \quad \lim_{\tau \rightarrow 0} \frac{o(\tau)}{\tau} = 0, \quad \lim_{\tau \rightarrow 0} \frac{o_k(\tau)}{\tau} = 0
}

\eci
}
\end{frame}

%%%%%%%%%%%%%%%%%%%%%%%%%%%%%%%%%%%%%%%%%%%%%%%%%%%%%%
\begin{frame}{Poisson Process: Definition (2)}

An arrival process is called a \bluef{Poisson process} with rate
$\lambda$, if the following are satisfied:
\mycolorbox{
\small
  \plitemsep 0.05in
\bci
\item \redf{(Independence)}
  Let $N_\tau$ be the number of arrivals over the interval $[0,\tau].$ For
  any $\tau_1, \tau_2 >0,$ $N_{s+\tau_1} - N_s$ is independent of
  $N_{t+\tau_2} - N_t,$ if $t > s+ \tau_1.$

\medskip

  $\circ$ \magenf{The number of arrivals over two disjoint intervals are independent.}

\item \redf{(Time homogeneity)} For any $s,$ the distribution of
  $N_{s+\tau} - N_s$ is equal to that of $N_\tau.$ 

\medskip

  $\circ$  \magenf{$N_\tau$ becomes the number of arrivals over \bluef{any} interval of
  length $\tau.$}

% \onslide<3->{- Thus, $N_s$ can be a random variable over \bluef{any} interval of length $s.$}

\item \bluef{(Distribution of $N_\tau$) $N_\tau$ is the Poisson rv
  with parameter $\lambda \tau,$ i.e., if we let $\cprob{k,\tau} =
  \cprob{N_\tau = k},$ we have:
  $$
  \cprob{k,\tau} = e^{\lambda \tau} \frac{(\lambda \tau)^k}{k!}, \quad
  k = 0, 1, 2, \ldots
  $$
  }
\eci
}
\end{frame}



%%%%%%%%%%%%%%%%%%%%%%%%%%%%%%%%%%%%%%%%%%%%%%%%%%%%%%
\begin{frame}{Poisson Process: $\cprob{k,\tau},$ $N_\tau,$ and $T$}

\plitemsep 0.1in
\bci
\item<2->[\redf{(Q1)}] Number of arrivals of \orangef{any interval} with length $\tau$ $\sim \text{Poisson}(\lambda \tau),$ i.e.,
\aleq{
\cprob{k,\tau} = e^{-\lambda \tau}\frac{(\lambda \tau)^k}{k!}, \quad k=0,1,2, \ldots
}

\item<3->[] $\cexpect{N_\tau} = \lambda \tau$ and $\cvar{N_\tau} = \lambda \tau$

\item<4->[\redf{(Q2)}] Time of first arrival $T$
\aleq{
\onslide<5->{F_T(t) &= \cprob{T \le t} = 1- \cprob{T >t} = 1- P(0,t) = 1- e^{-\lambda t}}\cr
\onslide<6->{f_T(t) &= \frac{dF_T(t)}{dt} = \lambda e^{-\lambda t}, \quad t \ge 0.}
}

\item<7-> $T \sim \text{Exp}(\lambda).$ Thus, $\cexpect{T} = 1/\lambda$ and $\cvar{T} = 1/\lambda^2$

- Continuous twin of geometric rv in Bernoulli process

- Memoryless


\eci
\end{frame}

%%%%%%%%%%%%%%%%%%%%%%%%%%%%%%%%%%%%%%%%%%%%%%%%%%%%%%
\begin{frame}{Poisson Process: Example}

\plitemsep 0.15in
\bci
\item Receive emails according to a Poisson process at rate $\lambda =5$ messages/hour

\item Mean and variance of mails received during a day

\onslide<2->{- 5* 24 = 120}

\item $\prob{\text{one new message in the next hour}}$

\onslide<3->{- $\displaystyle \cprob{1,1} = \frac{(5\cdot 1)^1 e^{-5 \cdot 1}}{1 !} = 5 e^{-5}$}

\item $\prob{\text{exactly two msgs during each of the next three hours}}$

\onslide<4->{- $\displaystyle \lf (\frac{5^2 e^{-5}}{2!} \ri)^3$}
\eci
\end{frame}

%%%%%%%%%%%%%%%%%%%%%%%%%%%%%%%%%%%%%%%%%%%%%%%%%%%%%%
\begin{frame}{Memoryless and Fresh-start Property}

\plitemsep 0.1in
\bci
\item \redf{Remind.} Similar property for Bernoulli processes, but here no time slots.

\item<2-> \redf{Fresh-start at determinsitic time $t$:} Start watching at time $t,$ then you see the Poisson process, independent of what has happened in the past.

\item<3-> \redf{Fresh-start at random time $T$:} Similarly, it
  holds.

  - For example, when you start watching at random time $T_1$
  (time of first arrival).

  - Generally, it holds when $T$ is a \bluef{stopping
  time}. 

\item<4->[\redf{(Q3)}] The $k$-th arrival time $Y_k$?

\item<5-> $k$-th inter-arrival time $T_k = Y_k - Y_{k-1},$ $k\ge 2,$ and $T_1 = Y_1.$

\item<6-> $Y_k = T_1 + T_2 + \cdots + T_k$ is sum of i.i.d. exponential rvs.

\item<7-> $\expect{Y_k} = k/\lambda$ and $\var{Y_k} = k/\lambda^2$,
  but what is the distribution of $Y_k$?
\eci
\end{frame}

%%%%%%%%%%%%%%%%%%%%%%%%%%%%%%%%%%%%%%%%%%%%%%%%%%%%%%
\begin{frame}{PDF of $Y_k$}

\plitemsep 0.04in
\bci
\item<2-> For a given $\delta$, \redblank{3}{$\delta\cdot f_{Y_k}(y)$}: prob. of $k$-th arrival over $[y,y+\delta].$

\item<4-> When $\delta$ is small, only one arrival occurs. Thus,
 \aleq{
 \delta\cdot f_{Y_k}(y) &= \onslide<5->{\cbprob{\text{an arrival over $[y,y+\delta]$}} \times \cbprob{\text{$k-1$ arrivals before $y$}}} \cr
 &\onslide<6->{\approx \lambda \delta \times \cprob{k-1,y} = \lambda \delta \times \frac{\lambda^{k-1}y^{k-1}e^{-\lambda y}}{(k-1)!}}\cr
  \onslide<7->{f_{Y_k}(y) &= \frac{\lambda^k y^{k-1}e^{-\lambda y}}{(k-1)!}, \quad y\ge 0.}
}

\item<8-> This is called \redf{Erlang} rv.

  \mycolorbox{
    An Erlang random variable $Z$ with parameter $(k,\lambda)$ has the
    following pdf:
    $$
    f_{Z}(z) = \frac{\lambda^k z^{k-1}e^{-\lambda z}}{(k-1)!}, \quad z\ge 0
    $$
  }
% \bigskip
% \item<9-> \bluef{Bernoulli process} vs. \orangef{Poisson process}
% \bci
% \item<9-> Time of first arrival: \bluef{geometric} / \orangef{exponential}
% \item<9-> Time of $k$-th arrivals: \bluef{Pascal} / \orangef{Erlang}
\eci
\end{frame}





%%%%%%%%%%%%%%%%%%%%%%%%%%%%%%%%%%%%%%%%%%%%%%%%%%%%%%
\begin{frame}{Poisson Process vs. Bernoulli Process}

- $n= \tau/\delta$, $p = \lambda \delta$, $np = \lambda \tau$

\vspace{-0.8cm}
\begin{center}
\mypic{0.45}{L8_tau_delta.jpg}
\end{center}

\begin{center}
\begin{tabular}{|c|c|c|} \hline
  & Bernoulli process & Poisson process \\ \hline \hline
  time of arrival & Discrete & Continuous \\ \hline
  PMF of \# of arrivals & \onslide<2->{Binomial} & \onslide<2->{Poisson} \\ \hline 
  Interarrival time & \onslide<3->{Geometric} & \onslide<3->{Exponential} \\ \hline
  Time of $k$-th arrival & \onslide<4->{Pascal} & \onslide<4->{Erlang} \\ \hline
  Arrival rate& \onslide<5->{$p$/per slot}& \onslide<5->{$\lambda$/unit time} \\ \hline 
\end{tabular}
\end{center}
% \vspace{-0.8cm}
% \centering
% \mypic{0.65}{L8_poisson_bernoulli.png}
\end{frame}

%%%%%%%%%%%%%%%%%%%%%%%%%%%%%%%%%%%%%%%%%%%%%%%%%%%%%%
\begin{frame}{Example: Poisson Fishing (Problem 10, page 329)}

\plitemsep 0.05in
\bci
\item Catching fish: Poisson process $\lambda = 0.6$/hour.
\item Fish for 2 hours. Within 2 hours, if he catches at least one fish, then he stops at 2 hours. Otherwise, he fishes until one fish is caught.
\eci

%\medskip
\mytwocols{0.55}
{
\small
\plitemsep 0.05in
\bce[\bf (Q1)]
\item<2-> $\cprob{\text{fishing time $>$ 2 hours}}$

\smallskip
\onslide<3->{Method 1: $\cprob{0,2}$}

\onslide<4->{Method 2: $\cprob{T_1 >2}$}

\item<5->  $\cprob{\text{2 $<$ fishing time $<$ 5}}$

\smallskip
\onslide<6->{Method 1: $\cprob{0,2}(1-\cprob{0,3})$}

\onslide<7->{Method 2:$\cprob{2 < T_1 <5}$}

\item<8-> $\cprob{\text{Catch at least two fish}}$

\smallskip
\onslide<9->{Method 1:$\sum_{k=2}^\infty = 1 - \cprob{0,2} - \cprob{1,2}$}

\onslide<10->{Method 2: $\cprob{Y_2 \le 2}$}

\ece
}
{
\small
\plitemsep 0.05in
\bce[\bf (Q4)]
\item<11-> $\expect{\text{future fi. time}|\text{already fished for 3h}}$

\onslide<12->{Fresh-start. So, $\expect{exp(\lambda)} = 1/\lambda = 1/0.6$}

\item<13->[\bf (Q5)] $\expect{$F=$\text{total fishing time}}$
\aleq{
\onslide<14->{2 + \expect{F-2} &= 2+ \cprob{F=2}\cdot 0 + \cr
& \cprob{F >2}\cdot \expect{F-2 | F >2}\cr
&= 2 + \cprob{0,2}\cdot \frac{1}{\lambda}}
}
\item<15->[\bf (Q6)] $\expect{\text{number of fish}}$ = $\lambda \cdot 2 + \cprob{0,2}\cdot 1$
\ece

}

\end{frame}


 %%%%%%%%%%%%%%%%%%%%%%%%%%%%%%%%%%%%%%%%%%%%%%%%%%%%%%
\section{L8(5)}
\begin{frame}{Roadmap}

\plitemsep 0.2in

\bce[(1)]

\item \grayf{Introduction of Random Processes}
\item \grayf{Bernoulli Processes}

\item \grayf{Poisson Processes: Poisson RV and Basic Idea}

\item \grayf{Poisson Processes: Definition and Properties}

\item \redf{Playing with Bernoulli and Poisson Processes}

\item \grayf{Random Incidence Phenomenon}

  \ece

\end{frame}

%%%%%%%%%%%%%%%%%%%%%%%%%%%%%%%%%%%%%%%%%%%%%%%%%%%%%%
\begin{frame}{Description via Inter-arrival Times}

%\vspace{-0.5cm}
% \smallskip
% \redblank{1}{\mypic{0.8}{L8_coding_bernoulli.png}}
\mycolorbox{
  \mypic{0.8}{L8_coding_bernoulli.png}
}
% \smallskip
% \redblank{1}{\mypic{0.8}{L8_coding_poisson.png}}
\mycolorbox{
  \mypic{0.8}{L8_coding_poisson.png}
}

\end{frame}

%%%%%%%%%%%%%%%%%%%%%%%%%%%%%%%%%%%%%%%%%%%%%%%%%%%%%%
\begin{frame}{Example}

  \plitemsep 0.1in

  \bci
  
\item It has been observed that after a rainy day, the number of days
  until the next rain $\sim \text{Geom}(p),$ independent of the past. 
  
\item<2-> \question What is the probability that it rains on both the 5th and the
  8th day of the month?

\item<3-> \bluef{Approach 1}: Handling this problem directly with the geometric
  PMFs is very tedious and complex.

\item<4-> \bluef{Approach 2}: Rainy days is a  Bernoulli process with arrival.
  probability $p.$

\item<5-> Thus, the answer is $p^2.$

  \eci

\end{frame}

%%%%%%%%%%%%%%%%%%%%%%%%%%%%%%%%%%%%%%%%%%%%%%%%%%%%%%
\begin{frame}{Coding of Random Arrivals}

  \plitemsep 0.1in

  \bci
  
\item \question How to make software codes of Bernoulli process with $p$ and Poisson process with $\lambda$

\item Inter-arrival times are very handy. 
  
\item \bluef{Bernoulli process with $p$}: Obtain a sequence of random values
  following the \bluef{geometric distribution} with parameter $p.$

\item \redf{Poisson process with $\lambda$}: Obtain a sequence of random values
  following the \redf{exponential distribution} with parameter $\lambda.$
  
  \eci

\end{frame}

%%%%%%%%%%%%%%%%%%%%%%%%%%%%%%%%%%%%%%%%%%%%%%%%%%%%%%
\begin{frame}{Notations In the Rest of These Slides}

  \plitemsep 0.2in

  \bci
  
\item Bernoulli random variable: \bluef{$\bern{p}$}

\item Bernoulli process: \redf{$\bp{p}$}

\item Poisson random variable: \bluef{$\poisson{\lambda}$}

\item Poisson process: \redf{$\pp{\lambda}$}

  \eci

\end{frame}


%%%%%%%%%%%%%%%%%%%%%%%%%%%%%%%%%%%%%%%%%%%%%%%%%%%%%%
\begin{frame}{Split: Bernoulli Process}

\plitemsep 0.05in
\bci
\item Split $\bp{p}$ into two processes. Whenever there is an arrival, keep it w.p. $q$ and
  discard it w.p. $(1-q).$

\item<2-> Split decisions are independent of arrivals.

\item<3-> \question What are the two split processes?


\item<4-> \redf{$\bp{pq}$} and \redf{$\bp{p(1-q)}$}. Why?

\item<5-> Are they independent? \magenf{No.} 
  \eci

%\medskip
\vspace{-0.7cm}
\mypic{0.5}{L8_split_bernoulli}

\end{frame}

%%%%%%%%%%%%%%%%%%%%%%%%%%%%%%%%%%%%%%%%%%%%%%%%%%%%%%
\begin{frame}{Merge: Bernoulli Process}

\plitemsep 0.05in
\bci
\item Merge $\bp{p}$ and $\bp{q}$ into one process.

\item<2-> Collided arrival is regarded just one arrival in the merged process

\item<3-> Probability of having at least one arrival: $1-(1-p)(1-q) = p+q-pq$
    

\item<4-> Merged process: \redf{$\bp{p+q - pq}$}

  \item<5-> $\cprob{\text{arrival from proc. 1} \mid \text{arrival in
        the merged proc.}} = \redf{\frac{p}{p+ q - pq}}$ 
  \eci

\vspace{-0.5cm}
\mypic{0.5}{L8_merge_bernoulli}


\end{frame}

%%%%%%%%%%%%%%%%%%%%%%%%%%%%%%%%%%%%%%%%%%%%%%%%%%%%%%
\begin{frame}{Split: Poisson Process}

\plitemsep 0.07in
\bci
\item Split a Poisson process $\pp{\lambda}$ into two processes by
  keeping each arrival w.p. $p$ and discarding it w.p. $(1-p)$

\item<2-> \question What are the split processes?

\item<3-> Let's focus on the process that we keep

\item<4-> Independence and time-homogeneity? \bluef{Yes}  
\item<5-> Small interval probability over $\delta$-interval
  \bci
  \item<6->$\cprob{\text{1 arrival}} = p \lambda \delta +
    p\cdot o(\delta) = p\lambda \delta + o(\delta)$
  \item<6-> $\cprob{\text{2 arrivals}} = p \cdot o(\delta) = o(\delta)$ 
  \item<6-> $\cprob{\text{0 arrival}} = 1 - p \lambda \delta -
    p\cdot o(\delta) - p \cdot o(\delta) = 1 - p\lambda \delta + o(\delta)$ 
    \eci

\item<6-> \redf{\pp{\lambda p}} and \redf{\pp{\lambda(1-p)}}     
\eci
\end{frame}



%%%%%%%%%%%%%%%%%%%%%%%%%%%%%%%%%%%%%%%%%%%%%%%%%%%%%%
\begin{frame}{Merge: Poisson Process (1)}

\plitemsep 0.1in
\bci

\item Merge from $\pp{\lambda_1}$ and $\pp{\lambda_2}$

\item<2-> Independence and time-homogeneity? \bluef{Yes}

\item<3-> Small interval probabilty over $\delta$-interval (ignoring
  $o(\delta)$ for small $\delta$)
\aleq{
\cprob{\text{0 arrival} } &\approx (1-\lambda_1
\delta)(1-\lambda_2 \delta) \approx 1-(\lambda_1 + \lambda_2)\delta \cr
\cprob{\text{1 arrival} }& \approx (\lambda_1
\delta)(1-\lambda_2 \delta) + \lambda_2 \delta (1-\lambda_1 \delta) \approx
(\lambda_1 + \lambda_2) \delta % \cr
% \cprob{\text{$\geq$ 2 arrivals} } & = o(\delta)
}

\item<4-> Merged process: \redf{$\pp{\lambda_1 + \lambda_2}$}
\eci
\end{frame}

%%%%%%%%%%%%%%%%%%%%%%%%%%%%%%%%%%%%%%%%%%%%%%%%%%%%%%
\begin{frame}{Merge: Poisson Process (2)}

\plitemsep 0.1in
\bci

\item \redf{Red}: $\pp{\lambda_1}$ and \bluef{Blue}: $\pp{\lambda_2}$

\item<2-> $\cprob{\text{from \redf{red}} \mid \text{arrival at time $t$ in the merged
      proc.}}$?  \onslide<3->{$\displaystyle \frac{\lambda_1\delta}{\lambda_1
    \delta+ \lambda_2 \delta}= \frac{\lambda_1}{\lambda_1
    + \lambda_2}$}
  
\item<4-> Consider an event $A_k = \{\text{$k$-th arrival in the merged
    proc. is \redf{red}} \}$
  \bci
 \item<5->   $\cprob{A_k}$? \onslide<6->{$\displaystyle \frac{\lambda_1}{\lambda_1 + \lambda_2}$}
\item<7-> $A_1, A_2, \ldots$ are independent: origins (red or blue) of arrivals in the
  merged proc. are indepdnent

  \eci

\item<8-> $\cprob{\text{k out of first 10 arrivals are red}}$?
\onslide<9->{
  $\displaystyle {10 \choose
  k} \left(\frac{\lambda_1}{\lambda_1 + \lambda_2}\right)^k
  \left(\frac{\lambda_2}{\lambda_1 + \lambda_2}\right)^{10-k}$ }
  
  \eci
\end{frame}

%%%%%%%%%%%%%%%%%%%%%%%%%%%%%%%%%%%%%%%%%%%%%%%%%%%%%%
\begin{frame}{Using Poisson Processes for Intuitive Problem Solving}
  
  \plitemsep 0.2in
  \bce
  
\item Competing exponentials
  
\item Sum of independent Poisson rvs
  
\item Poisson arrivals during and Exponential interval

%\item Random incidence: Impact of different sampling methods

  \ece
\end{frame}

%%%%%%%%%%%%%%%%%%%%%%%%%%%%%%%%%%%%%%%%%%%%%%%%%%%%%%
\begin{frame}{Competing Exponential (1)}

\plitemsep 0.1in
\bci
\item Two independent light bulbs have life times $T_a \sim \text{Exp}(\lambda_a)$ and $T_b \sim \text{Exp}(\lambda_b).$

\item<2->[\redf{(Q)}] Distribution of $Z = \min\{T_a, T_b \},$ the
  first time when a bulb burns out?


\item<3-> \bluef{Approach 1}
  \bci
\item<3-> $\cprob{Z \geq z} = \cprob{T_a \geq
    z}\cprob{T_b \geq z} = e^{-\lambda_a z} e^{-\lambda_b z}=  e^{-(\lambda_a+ \lambda_b) z}$
\item<4-> Thus, $Z \sim \text{Exp}(\lambda_a + \lambda_b)$
  \eci

\item<5-> \bluef{Approach 2}

\bci
\item<5-> $T_a$ and $T_b$ are the first arrival times of two Poisson
  processes of $\lambda_a$ and $\lambda_b,$ respectively. 

\item<6-> $Z$ is the first arrival time of merged Poisson process
  $(\lambda_a + \lambda_b).$

\item<7-> Thus, $Z \sim \text{Exp}(\lambda_a + \lambda_b)$
\eci
\eci

\end{frame}

%%%%%%%%%%%%%%%%%%%%%%%%%%%%%%%%%%%%%%%%%%%%%%%%%%%%%%
\begin{frame}{Competing Exponential (2)}
  
  \plitemsep 0.1in
  \bci
\item Three independent light bulbs have life time $T \sim \text{Exp}(\lambda).$
  
\item[\redf{(Q)}] $\expect{\text{time until the last bulb burns out}}$?
  
  
\item<2-> Understanding from the merged Poisson process
  
  \bci
\item<3->   $T_1$: time until the first burn-out, $T_2$: time until the second burn-out, $T_3$: time until the third burn-out

  \item<3-> $\cexpect{T_1 + T_2 + T_3}$?
  
\item<4-> $\pp{3\lambda}$ $\xrightarrow{\text{1st burn out}}$
  $\pp{2\lambda}$ $\xrightarrow{\text{2st burn out}}$
  $\pp{\lambda}$
  
  
\item<4-> $T_1 \sim \text{Exp}(3\lambda)$, $T_2 \sim \text{Exp}(2\lambda)$, $T_3 \sim \text{Exp}(\lambda)$
\onslide<5->{
  $$
  \bexpect{T_1 + T_2 + T_3}= \frac{1}{3 \lambda} + \frac{1}{2 \lambda} + \frac{1}{ \lambda} $$
}
  \eci
  \eci
  
\end{frame}


%%%%%%%%%%%%%%%%%%%%%%%%%%%%%%%%%%%%%%%%%%%%%%%%%%%%%%

\begin{frame}{Sum of Independent Poisson rvs}

\plitemsep 0.1in

\bci

\item Two independent rvs $X$ and $Y$, where $X \sim \poisson{\mu}$ and  $Y \sim \poisson{\nu}.$

\item<2-> \question Distribution of $X+Y$? Complex convolution, but any other easy way?

\item<3-> Poisson process perspective
  \bci
  \item $X$: number of arrivals of $\pp{1}$ over a time interval of length $\mu$
  \item $Y$: number of arrivals of $\pp{1}$ over a time interval of length $\nu$
  \item Two intervals do not overlap and located in a consecutive
    manner $\implies$ $X \indep Y$
  \eci
  
\item<4-> Distribution of $X+Y$: the number of arrivals of $\pp{1}$ over a
  time interval of length $\mu + \nu$ 

\item<5-> Thus, \bluef{$ X+ Y \sim \poisson{\mu + \nu}$}

\eci


\end{frame}

%%%%%%%%%%%%%%%%%%%%%%%%%%%%%%%%%%%%%%%%%%%%%%%%%%%%%%

\begin{frame}{Poisson arrivals during Exponential Interval (1)}

\plitemsep 0.1in

\bci

\item Problem 24, pp. 335

\item Consider $\pp{\lambda}$ and an independent rv $T \sim \text{Exp}(\nu)$

\item<2-> \question Distribution of $N_T$?

\item<3-> \bluef{Approach 1}: Total probability theorem

\aleq
  {
    \cprob{N_T = k} = \onslide<4->{\int_{0}^\infty\cprob{N_T = k | T = \tau}
    f_{T}(\tau) d\tau = \int_{0}^\infty \cprob{N_\tau = k}f_{T}(\tau) d\tau}
  }
  
\item<5-> Very tedious and not very intuitive. 

  \eci


\end{frame}

%%%%%%%%%%%%%%%%%%%%%%%%%%%%%%%%%%%%%%%%%%%%%%%%%%%%%%

\begin{frame}{Poisson arrivals during Exponential Interval (2)}

\plitemsep 0.1in

\bci

\item Consider $\pp{\lambda}$ and an independent rv $T \sim \text{Exp}(\nu)$
\item<2-> Consider another $\pp{\nu},$ and let's view $T$ as the first
  arrival time in $\pp{\nu}.$

\item<3-> Now, consider the merged process of $\pp{\lambda}$ and
  $\pp{\nu}.$
  \bci
\item $\bprob{\text{from $\pp{\lambda}$}| \text{arrival}} =
  \frac{\lambda}{\lambda+\nu}$ and $\bprob{\text{from $\pp{\nu}$}| \text{arrival}} =
  \frac{\nu}{\lambda+\nu}$ 
  \eci
\item<4-> $K$: number of total arrivals until we get the first arrival from
  $\pp{\nu}.$
  \bci
\item<5->   Then, $K \sim \text{Geom}(\frac{\nu}{\lambda+\nu}).$

  \eci


\item<6-> Let $L$ be the number of arrivals from $\pp{\lambda}$
  until we get the first arrival from $\pp{\nu}.$
  \aleq{
    p_L(l) = p_K(l+1) = \left(\frac{\nu}{\lambda + \nu}\right)
    \left(\frac{\lambda}{\lambda + \nu} \right)^l, \quad l = 0, 1, \ldots
  }

  \eci


\end{frame}


% %%%%%%%%%%%%%%%%%%%%%%%%%%%%%%%%%%%%%%%%%%%%%%%%%%%%%%
% \begin{frame}{Other Interesting Topics}
  
%   \plitemsep 0.2in
%   \bce
  
% \item Random incidence: Impact of different sampling methods

% \item Approximation of Binomial: Poisson vs. Normal  

%   \ece
% \end{frame}


%%%%%%%%%%%%%%%%%%%%%%%%%%%%%%%%%%%%%%%%%%%%%%%%%%%%%% 
\begin{frame}{Approximation of Binomial: Poisson vs. Normal}
  
  \plitemsep 0.1in
  \bci
  
\item Consider $S_n \sim \text{Binomial}(n,p).$ Then, $S_n =
  \sum_{i=1}^n X_i,$ where $X_i \sim \bern{p}.$
\item<2-> \bluef{Poisson approximation.} $\poisson{np}$ is a good approximation of $S_n$
  \bci
  \item<4-> Holds when $np = \lambda$ ($n \rightarrow \infty$, $p
    \rightarrow 0$), i.e., $X_i$'s behavior changes over $n.$
  \eci
\item<3-> \bluef{Normal approximation.} $\sigma\sqrt{n}Z + n\mu$ is a good approximation of $S_n$  
  \bci
  \item<5-> Holds for a fixed $p$ ($n \rightarrow \infty$), i.e., $X_i$'s
    behavior does not change over $n.$
  \eci
\item<6-> In practice, an actual numbers of $n$ and $p$ are given, so
  which approximation is good under what situation? 
  \bci
  \item<7-> $p=1/100$, $n=100$: $np=1$, very asymmetric $X_i$, small $p$
    $\implies$ \orangef{Poisson}
  \item<8-> $p=1/3$, $n=100$: large, reasonly symmetric $p,$ at least moderate $n$ $\implies$ \orangef{Normal}

  \item<9-> $p=1/100$, $n=10,000$: small $p$, but large $n$ $\implies$ \orangef{Both Poisson
    and Normal} 
    \eci
  \eci
\end{frame}


 %%%%%%%%%%%%%%%%%%%%%%%%%%%%%%%%%%%%%%%%%%%%%%%%%%%%%%
\section{L8(6)}
\begin{frame}{Roadmap}

\plitemsep 0.2in

\bce[(1)]

\item \grayf{Introduction of Random Processes}
\item \grayf{Bernoulli Processes}

\item \grayf{Poisson Processes: Poisson RV and Basic Idea}

\item \grayf{Poisson Processes: Definition and Properties}

\item \grayf{Playing with Bernoulli and Poisson Processes}

\item \redf{Random Incidence Phenomenon}

  \ece

\end{frame}

%%%%%%%%%%%%%%%%%%%%%%%%%%%%%%%%%%%%%%%%%%%%%%%%%%%%%%
\begin{frame}{Example: Survey of Utilization of Town Buses}

\plitemsep 0.15in

\bci

\item<1-> What we want to survey: How available are town buses in a city?

\item<2-> Two approaches

  \bce[M1.]
  \item<3-> (1) select a few buses at random, and (2) calculate the
    average number of riders in the selected bus

  \item<4-> (1) select a few bus riders at random, (2) look at the buses
    that they rode, and (3) calculate the average number of riders in
    those buses
    \ece

\item<5-> Which is correct?

  \item<6-> (i) M1 $=$ M2? (ii) M1 $>$ M2? (iii) M1 $<$ M2?    



  
% \item Answer: M1 $<$ M2

% \item \redf{Why?} We will discuss this later.
%More likely to select a bus with a large number of
%  riders than a bus that is near-empty. 
  \eci

\end{frame}

%%%%%%%%%%%%%%%%%%%%%%%%%%%%%%%%%%%%%%%%%%%%%%%%%%%%%%
\begin{frame}{Random Incidence Phenomemon (1)}

\plitemsep 0.1in

\bci

\item \redf{We know:} in $\pp{\lambda}$, inter-arrival time $\sim \text{Exp}(\lambda)$ 

\item<2-> Fix a time instant $t^\star,$ and consider the length $L$ of the
  inter-arrival interval that \redf{constains $t^\star.$}

  \onslide<2->{\mypic{0.45}{L8_random_incidence.png}}

\item<3-> \bluef{Practical context:} Yung shows up at the bus station at some
  arbitrary time $t^\star$ and records the time from the previous bus
  arrival ($U$) until the next bus arrival ($V$)

\item<4-> \question \bluef{What is the distribution of $L$?} \hfill \lecturemark{VIDEO PAUSE}
  
  \eci


\end{frame}

%%%%%%%%%%%%%%%%%%%%%%%%%%%%%%%%%%%%%%%%%%%%%%%%%%%%%%
\begin{frame}{Random Incidence Phenomemon (2)}

\plitemsep 0.1in

  \mypic{0.45}{L8_random_incidence.png}

\bci

\item<2-> $t^\star$ is \redf{not} random, so ``random incidence'' may be
  confusing and misleading. 

  \item<3-> \redf{Assumption.} For simplicity, $t^\star$ is large enough that we must have
    an arrival before $t^\star$ ($U > 0$)

\item<4-> One might superficially argue that $L \sim
  \text{Exp}(\lambda),$ but it is \redf{NOT}. 
    \eci

\end{frame}

%%%%%%%%%%%%%%%%%%%%%%%%%%%%%%%%%%%%%%%%%%%%%%%%%%%%%%
\begin{frame}{Distribution of $L$ (1)}

\plitemsep 0.1in

\begin{center}
\mysizebox{0.37}{$L = (t^\star - U) + (V - t^\star)$} 
\end{center}

\myvartwocols{0.55}{0.68}{0.3}
{
\small
  \bci

\item $V - t^\star$: \onslide<2->{Due to the \bluef{memoryless} property and the
  \bluef{fresh-restart},}  
  \aleq{
    \onslide<3->{\text{Thus,} \quad V- t^\star \sim \redf{\text{Exp}(\lambda)}}
  }
  
\item $t^\star - U$: \onslide<4->{If we run the $\pp{\lambda}$ backwards in time,
  it remains Poisson. Why?} \onslide<5->{More formally,} 
  \aleq{
    \onslide<5->{&\cprob{t^\star - U > x} = \cprob{\text{no arrivals over $[t^\star-x,t^\star]$}}\cr
    & = e^{-\lambda x} = \cprob{T_{\text{inter}} > x}}
  }
  \vspace{-0.9cm}
  \aleq{
    \onslide<6->{\text{Thus,} \quad t^\star - U \sim \redf{\text{Exp}(\lambda)}}
  }

  \eci
}
{
\mypic{0.99}{L8_random_incidence.png}
}

  
\end{frame}

%%%%%%%%%%%%%%%%%%%%%%%%%%%%%%%%%%%%%%%%%%%%%%%%%%%%%%
\begin{frame}{Distribution of $L$ (2)}

\plitemsep 0.1in

\begin{center}
\mysizebox{0.37}{$L = (t^\star - U) + (V - t^\star)$} 
\end{center}

\bci
\item<2-> $L = X_1 + X_2,$ where $X_1, X_2 \sim \text{Exp}(\lambda)$
\item<3-> Time until we have two arrivals in $\pp{\lambda}$
  
  \item<4-> Erlang random variable with parameter $(2,\lambda)$, i.e.,
    \hfill \lecturemark{page 37}
    $$
    f_{L}(l) = \lambda^2 \cdot l \cdot e^{-\lambda l}, \quad l\ge 0
    $$
\item<5-> Mean = $2/\lambda$    

  % \mycolorbox{
  %   An Erlang random variable $Z$ with parameter $(k,\lambda)$ has the
  %   following pdf:
  %   $$
  %   f_{Z}(z) = \frac{\lambda^k z^{k-1}e^{-\lambda z}}{(k-1)!}, \quad z\ge 0
  %   $$
  % }

\item<6-> Why not $\text{Exp}(\lambda)$? \onslide<7->{\bluef{An observer who arrives at an
  arbitrary time is more likely to fall in a large rather than a small
  interarrival interval.}}
  \eci
  
\end{frame}


%%%%%%%%%%%%%%%%%%%%%%%%%%%%%%%%%%%%%%%%%%%%%%%%%%%%%%
\begin{frame}{Back to Survey of Utilization of Town Buses}

\plitemsep 0.1in

\bci

\item Two Approaches

  \bce[M1.]
  \item (1) select a few buses at random, and (2) calculate the
    average number of riders in the selected bus

  \item (1) select a few bus riders at random, (2) look at the buses
    that they rode, and (3) calculate the average number of riders in
    those buses
    \ece

\item (i) M1 $=$ M2? (ii) M1 $>$ M2? (iii) M1 $<$ M2?    

\item Answer: M1 $<$ M2

\item \redf{More likely to select a bus with a large number of
 riders than a bus that is near-empty.}
  \eci

\end{frame}



%%%%%%%%%%%%%%%%%%%%%%%%%%%%%%%%%%%%%%%%%%%%%%%%%%%%%%

\begin{frame}{}
\vspace{2cm}
\LARGE Questions?

\end{frame}

%%%%%%%%%%%%%%%%%%%%%%%%%%%%%%%%%%%%%%%%%%%%%%%%%%%%%%
\begin{frame}{Review Questions}

\bce[1)]
\item Explain what is random process. Why do we need such a concept?

\item Explain Bernoulli processes. When do we use Bernoulli processes?

\item Explain Poisson processes. When do we use Poisson processes?

\item What are the relations between those two processces? What features
  do they share?

  \item In both processces, ho do we compute (i) number of arrivals
    over a given interval of time, (ii) time until the first arrival,
    (iii) time until $k$-th arrival?

  \item What is Pascal rvs and Erlang rvs?
    
  \item What is the ``stopping time'' and how is it related to
    fresh-restart?

\item See many examples on how merging and splitting of BP and PP
  can be used for intuitive soloving of many problems.     

  \ece

\end{frame}




\end{document}

%%% Local Variables:
%%% mode: latex
%%% TeX-master: t
%%% End:
