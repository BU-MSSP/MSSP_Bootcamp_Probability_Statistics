
\def\slidemode{
  \documentclass[fleqn,aspectratio=169]{beamer}
\usepackage{pgfpages}
}
\def\handoutmode{
  \documentclass[handout,fleqn,aspectratio=169]{beamer}
\usepackage{pgfpages}
\pgfpagesuselayout{resize to}[a4paper,landscape,border shrink=5mm]
}


%\def\pdfmode{handoutmode}
\def\pdfmode{slidemode}

\csname\pdfmode\endcsname

\mode<presentation>
{
  \usetheme{default}
  \usecolortheme{default}
  \usefonttheme{default}
  \setbeamertemplate{navigation symbols}{}
  \setbeamertemplate{caption}[numbered]
  \setbeamertemplate{footline}[frame number]  % or "page number"
  \setbeamercolor{frametitle}{fg=white}
  \setbeamercolor{footline}{fg=black}
}


\usepackage[english]{babel}
%\usepackage[utf8x]{inputenc}
\usepackage{tikz}
\usepackage{courier}
\usepackage{array}
\usepackage{bold-extra}
%\usepackage{minted}
%\usepackage[thicklines]{cancel}
%\usepackage{fancyvrb}
\usepackage{kotex}
\usepackage{paralist}
\usepackage{collectbox}
\usepackage{amsmath}
\usepackage{mathtools}
\usepackage{nccmath}
\usepackage{bm}
\usepackage{booktabs}
\usepackage{tcolorbox}
\usepackage[absolute,overlay]{textpos}

\xdefinecolor{dianablue}{rgb}{0.18,0.24,0.31}
\xdefinecolor{darkblue}{rgb}{0.1,0.1,0.7}
\xdefinecolor{darkgreen}{rgb}{0,0.5,0}
\xdefinecolor{darkgrey}{rgb}{0.35,0.35,0.35}
\xdefinecolor{darkorange}{rgb}{0.8,0.5,0}
\xdefinecolor{darkred}{rgb}{0.7,0,0}
\definecolor{darkgreen}{rgb}{0,0.6,0}
\definecolor{mauve}{rgb}{0.58,0,0.82}


\title[]{Lecture 5: Random Variable, Part III}
\author{Yi, Yung (이융)}
\institute{EE210: Probability and Introductory Random Processes\\ KAIST EE}
\date{\today}



%%%%%%%%%%%% real, integer notation
\newcommand{\real}{{\mathbb R}}
\newcommand{\integer}{{\mathbb Z}}

%%% set, vector, matrix
\newcommand{\set}[1]{\ensuremath{\mathcal #1}}
\renewcommand{\vec}[1]{\bm{#1}}
\newcommand{\mat}[1]{\bm{#1}}


%%% big parenthesis
\def\Bl{\Bigl}
\def\Br{\Bigr}
\def\lf{\left}
\def\ri{\right}


%%% floor notations
\newcommand{\lfl}{{\lfloor}}
\newcommand{\rfl}{{\rfloor}}
\newcommand{\floor}[1]{{\lfloor #1 \rfloor}}

%%% gradient
\newcommand{\grad}[1]{\nabla #1}

%%% definition
\newcommand{\eqdef}{\ensuremath{\triangleq}}
%%% imply
\newcommand{\imp}{\Longrightarrow}



\newcommand{\separator}{
%  \begin{center}
    \par\noindent\rule{\columnwidth}{0.3mm}
%  \end{center}
}

\newcommand{\mynote}[1]{{\it \color{red} [#1]}}







%%% equation alignment
\newcommand{\aleq}[1]{\begin{align*}#1\end{align*}}

%%%%%%%%%%%%%%%% colored emphasized font, blanked words

\newcommand{\empr}[1]{{\color{red}\emph{#1}}}
\newcommand{\empb}[1]{{\color{blue}\emph{#1}}}

%normal colored text
\newcommand{\redf}[1]{{\color{red} #1}}
\newcommand{\yellowf}[1]{{\color{yellow} #1}}
\newcommand{\bluef}[1]{{\color{blue} #1}}
\newcommand{\grayf}[1]{{\color{gray} #1}}
\newcommand{\magenf}[1]{{\color{magenta} #1}}
\newcommand{\greenf}[1]{{\color{green} #1}}
\newcommand{\cyanf}[1]{{\color{cyan} #1}}
\newcommand{\orangef}[1]{{\color{orange} #1}}


\newcommand{\blk}[1]{\underline{\mbox{\hspace{#1}}}}


\newcommand{\redblk}[1]{\framebox{\color{red} #1}}
\newcommand{\redblank}[2]{\framebox{\onslide<#1->{\color{red} #2}}}
\newcommand{\blueblk}[1]{\framebox{\color{blue} #1}}
\newcommand{\blueblank}[2]{\framebox{\onslide<#1->{\color{blue} #2}}}



\makeatletter
\newcommand{\mybox}{%
    \collectbox{%
        \setlength{\fboxsep}{1pt}%
        \fbox{\BOXCONTENT}%
    }%
}
\makeatother

\makeatletter
\newcommand{\lecturemark}{%
    \collectbox{%
        \setlength{\fboxsep}{1pt}%
        \fcolorbox{red}{yellow}{\BOXCONTENT}%
    }%
}
\makeatother

\usepackage{tcolorbox}
\newcommand{\mycolorbox}[1]{
\begin{tcolorbox}[colback=red!5!white,colframe=red!75!black]
#1
\end{tcolorbox}
}

%%%% figure inclusion
\newcommand{\mypic}[2]{
\begin{center}
\includegraphics[width=#1\textwidth]{#2}
\end{center}
}

\newcommand{\myinlinepic}[2]{
\makebox[0cm][r]{\raisebox{-4ex}{\includegraphics[height=#1]{#2}}}
}


%%%% itemized and enumerated list
\newcommand{\bci}{\begin{compactitem}}
\newcommand{\eci}{\end{compactitem}}
\newcommand{\bce}{\begin{compactenum}}
\newcommand{\ece}{\end{compactenum}}


%%%% making 0.5/0.5 two columns
%%%% how to use: first number: length of separation bar
% \mytwocols{0.6}
% {
% contents in the left column
% }
% {
% contents in the right column
% }
%%%%
\newcommand{\mytwocols}[3]{
\begin{columns}[T] \column{.499\textwidth} #2 \column{.001\textwidth} \rule{.3mm}{{#1}\textheight} \column{.499\textwidth} #3 \end{columns}}

\newcommand{\mythreecols}[4]{
\begin{columns}[T] \column{.31\textwidth} #2 \column{.001\textwidth} \rule{.3mm}{{#1}\textheight} \column{.31\textwidth} #3 \column{.001\textwidth} \rule{.3mm}{{#1}\textheight} \column{.31\textwidth} #4  \end{columns}}

\newcommand{\mysmalltwocols}[3]{
\begin{columns}[T] \column{.4\textwidth} #2 \column{.001\textwidth} \rule{.3mm}{{#1}\textheight} \column{.4\textwidth} #3 \end{columns}}

%%%% making two columns with customized ratios
%%%% how to use:
%first parameter: length of separation bar
%second parameter: ratio of left column
%third parameter: ratio of right column
% \mytwocols{0.6}{0.7}{0.29}
% {
% contents in the left column
% }
% {
% contents in the right column
% }
%%%%
\newcommand{\myvartwocols}[5]{
\begin{columns}[T] \column{#2\textwidth} {#4} \column{.01\textwidth} \rule{.3mm}{{#1}\textheight} \column{#3\textwidth} {#5} \end{columns}}

%%% making my block in beamer
%%% first parameter: title of block
%%% second parameter: contents of block
\newcommand{\myblock}[2]{
\begin{block}{#1} {#2}  \end{block}}

%%% independence notation
\newcommand{\indep}{\perp \!\!\! \perp}

%%%% probability with different shapes (parenthesis or bracket) and different sizes
%%% `i' enables us to insert the subscript to the probability
\newcommand{\bprob}[1]{\mathbb{P}\Bl[ #1 \Br]}
\newcommand{\prob}[1]{\mathbb{P}[ #1 ]}
\newcommand{\cbprob}[1]{\mathbb{P}\Bl( #1 \Br)}
\newcommand{\cprob}[1]{\mathbb{P}( #1 )}
\newcommand{\probi}[2]{\mathbb{P}_{#1}[ #2 ]}
\newcommand{\bprobi}[2]{\mathbb{P}_{#1}\Bl[ #2 \Br]}
\newcommand{\cprobi}[2]{\mathbb{P}_{#1}( #2 )}
\newcommand{\cbprobi}[2]{\mathbb{P}_{#1}\Bl( #2 \Br)}

%%%% expectation with different shapes (parenthesis or bracket) and different sizes
%%% `i' enables us to insert the subscript to the expectation
\newcommand{\expect}[1]{\mathbb{E}[ #1 ]}
\newcommand{\cexpect}[1]{\mathbb{E}( #1 )}
\newcommand{\bexpect}[1]{\mathbb{E}\Bl[ #1 \Br]}
\newcommand{\cbexpect}[1]{\mathbb{E}\Bl( #1 \Br)}
\newcommand{\bbexpect}[1]{\mathbb{E}\lf[ #1 \ri]}
\newcommand{\expecti}[2]{\mathbb{E}_{#1}[ #2 ]}
\newcommand{\bexpecti}[2]{\mathbb{E}_{#1}\Bl[ #2 \Br]}
\newcommand{\bbexpecti}[2]{\mathbb{E}_{#1}\lf[ #2 \ri]}

%%%% variance
\newcommand{\var}[1]{\text{var}[ #1 ]}
\newcommand{\bvar}[1]{\text{var}\Bl[ #1 \Br]}
\newcommand{\cvar}[1]{\text{var}( #1 )}
\newcommand{\cbvar}[1]{\text{var}\Bl( #1 \Br)}

%%%% covariance
\newcommand{\cov}[1]{\text{cov}( #1 )}
\newcommand{\bcov}[1]{\text{cov}\Bl( #1 \Br)}

%%% Popular pmf, pdf notation to avoid long typing
\newcommand{\px}{\ensuremath{p_X(x)}}
\newcommand{\py}{\ensuremath{p_Y(y)}}
\newcommand{\pz}{\ensuremath{p_Z(z)}}
\newcommand{\pxA}{\ensuremath{p_{X|A}(x)}}
\newcommand{\pyA}{\ensuremath{p_{Y|A}(y)}}
\newcommand{\pzA}{\ensuremath{p_{Z|A}(z)}}
\newcommand{\pxy}{\ensuremath{p_{X,Y}(x,y)}}
\newcommand{\pxcy}{\ensuremath{p_{X|Y}(x|y)}}
\newcommand{\pycx}{\ensuremath{p_{Y|X}(y|x)}}

\newcommand{\fx}{\ensuremath{f_X(x)}}
\newcommand{\Fx}{\ensuremath{F_X(x)}}
\newcommand{\fy}{\ensuremath{f_Y(y)}}
\newcommand{\Fy}{\ensuremath{F_Y(y)}}
\newcommand{\fz}{\ensuremath{f_Z(z)}}
\newcommand{\Fz}{\ensuremath{F_Z(z)}}
\newcommand{\fxA}{\ensuremath{f_{X|A}(x)}}
\newcommand{\fyA}{\ensuremath{f_{Y|A}(y)}}
\newcommand{\fzA}{\ensuremath{f_{Z|A}(z)}}
\newcommand{\fxy}{\ensuremath{f_{X,Y}(x,y)}}
\newcommand{\Fxy}{\ensuremath{F_{X,Y}(x,y)}}
\newcommand{\fxcy}{\ensuremath{f_{X|Y}(x|y)}}
\newcommand{\fycx}{\ensuremath{f_{Y|X}(y|x)}}

\newcommand{\fth}{\ensuremath{f_\Theta(\theta)}}
\newcommand{\fxcth}{\ensuremath{f_{X|\Theta}(x|\theta)}}
\newcommand{\fthcx}{\ensuremath{f_{\Theta|X}(\theta|x)}}

\newcommand{\pkcth}{\ensuremath{p_{X|\Theta}(k|\theta)}}
\newcommand{\fthck}{\ensuremath{f_{\Theta|X}(\theta|k)}}


%%%% indicator
\newcommand{\indi}[1]{\mathbf{1}_{ #1 }}

%%%% exponential rv.
\newcommand{\elambdax}{\ensuremath{e^{-\lambda x}}}

%%%% normal  rv.
\newcommand{\stdnormal}{\ensuremath{\frac{1}{\sqrt{2\pi}} e^{-x^2/2}}}
\newcommand{\gennormal}{\ensuremath{\frac{1}{\sigma\sqrt{2\pi}} e^{-(x-\mu)^2/2}}}

%%%%%% estimator, estimate
\newcommand{\hth}{\ensuremath{\hat{\theta}}}
\newcommand{\hTH}{\ensuremath{\hat{\Theta}}}
\newcommand{\MAP}{\ensuremath{\text{MAP}}}
\newcommand{\LMS}{\ensuremath{\text{LMS}}}
\newcommand{\LLMS}{\ensuremath{\text{L}}}
\newcommand{\ML}{\ensuremath{\text{ML}}}

%%%% colored text
\newcommand{\red}[1]{\color{red}#1}
\newcommand{\cyan}[1]{\color{cyan}#1}
\newcommand{\magenta}[1]{\color{magenta}#1}
\newcommand{\blue}[1]{\color{blue}#1}
\newcommand{\green}[1]{\color{green}#1}
\newcommand{\white}[1]{\color{white}#1}

\newcommand{\defi}{{\color{red} Definition.} }
\newcommand{\proff}{{\color{red} Proof.} }
\newcommand{\exam}{{\color{red} Example.} }
\newcommand{\question}{{\color{red} Question.} }
\newcommand{\thm}{{\color{red} Theorem.} }
\newcommand{\background}{{\color{red} Background.} }
\newcommand{\msg}{{\color{red} Message.} }

\newcommand{\bern}[1]{\ensuremath{\text{Bern}(#1)} }
\newcommand{\bp}[1]{\ensuremath{\text{BP}(#1)} }
\newcommand{\poisson}[1]{\ensuremath{\text{Poisson}(#1)} }
\newcommand{\pp}[1]{\ensuremath{\text{PP}(#1)} }


%%%%%%%%%%%%%%%%%%%%%%% old macros that you can ignore %%%%%%%%%%%%%%%%%%%%%%%%

% \def\un{\underline}
% \def\ov{\overline}


% \newcommand{\beq}{\begin{eqnarray*}}
% \newcommand{\eeq}{\end{eqnarray*}}
% \newcommand{\beqn}{\begin{eqnarray}}
% \newcommand{\eeqn}{\end{eqnarray}}
% \newcommand{\bemn}{\begin{multiline}}
% \newcommand{\eemn}{\end{multiline}}
% \newcommand{\beal}{\begin{align}}
% \newcommand{\eeal}{\end{align}}
% \newcommand{\beas}{\begin{align*}}
% \newcommand{\eeas}{\end{align*}}



% \newcommand{\bd}{\begin{displaymath}}
% \newcommand{\ed}{\end{displaymath}}
% \newcommand{\bee}{\begin{equation}}
% \newcommand{\eee}{\end{equation}}


% \newcommand{\vs}{\vspace{0.2in}}
% \newcommand{\hs}{\hspace{0.5in}}
% \newcommand{\el}{\end{flushleft}}
% \newcommand{\bl}{\begin{flushleft}}
% \newcommand{\bc}{\begin{center}}
% \newcommand{\ec}{\end{center}}
% \newcommand{\remove}[1]{}

% \newtheorem{theorem}{Theorem}
% \newtheorem{corollary}{Corollary}
% \newtheorem{prop}{Proposition}
% \newtheorem{lemma}{Lemma}
% \newtheorem{defi}{Definition}
% \newtheorem{assum}{Assumption}
% \newtheorem{example}{Example}
% \newtheorem{property}{Property}
% \newtheorem{remark}{Remark}

% \newcommand{\separator}{
%   \begin{center}
%     \rule{\columnwidth}{0.3mm}
%   \end{center}
% }

% \newenvironment{separation}
% { \vspace{-0.3cm}
%   \separator
%   \vspace{-0.25cm}
% }
% {
%   \vspace{-0.5cm}
%   \separator
%   \vspace{-0.15cm}
% }

% \def\A{\mathcal A}
% \def\oA{\overline{\mathcal A}}
% \def\S{\mathcal S}
% \def\D{\mathcal D}
% \def\eff{{\rm Eff}}
% \def\bD{\bm{D}}
% \def\cU{{\cal U}}
% \def\bbs{{\mathbb{s}}}
% \def\bbS{{\mathbb{S} }}
% \def\cM{{\cal M}}
% \def\bV{{\bm{V}}}
% \def\cH{{\cal H}}
% \def\ch{{\cal h}}
% \def\cR{{\cal R}}
% \def\cV{{\cal V}}
% \def\cA{{\cal A}}
% \def\cX{{\cal X}}
% \def\cN{{\cal N}}
% \def\cJ{{\cal J}}
% \def\cK{{\cal K}}
% \def\cL{{\cal L}}
% \def\cI{{\cal I}}
% \def\cY{{\cal Y}}
% \def\cZ{{\cal Z}}
% \def\cC{{\cal C}}
% \def\cR{{\cal R}}
% \def\id{{\rm Id}}
% \def\st{{\rm st}}
% \def\cF{{\cal F}}
% \def\bz{{\bm z}}
% \def\cG{{\cal G}}
% \def\N{\mathbb{N}}
% \def\bbh{\mathbb{h}}
% \def\bbH{\mathbb{H}}
% \def\bbi{\mathbb{i}}
% \def\bbI{\mathbb{I}}
% \def\R{\mathbb{R}}
% \def\bbR{\mathbb{R}}
% \def\bbr{\mathbb{r}}
% \def\cB{{\cal B}}
% \def\cP{{\cal P}}
% \def\cS{{\cal S}}
% \def\bW{{\bm W}}
% \def\bc{{\bm c}}

% %\def\and{\quad\mbox{and}\quad}
% \def\ind{{\bf 1}}


% \def\bmg{{\bm{\gamma}}}
% \def\bmr{{\bm{\rho}}}
% \def\bmq{{\bm{q}}}
% \def\bmt{{\bm{\tau}}}
% \def\bmn{{\bm{n}}}
% \def\bmcapn{{\bm{N}}}
% \def\bmrho{{\bm{\rho}}}

% \def\igam{\underline{\gamma}(\lambda)}
% \def\sgam{\overline{\gamma}(\lambda)}
% \def\ovt{\overline{\theta}}
% \def\ovT{\overline{\Theta}}
% \def\PP{{\mathrm P}}
% \def\EE{{\mathrm E}}
% \def\iskip{{\vskip -0.4cm}}
% \def\siskip{{\vskip -0.2cm}}

% \def\bp{\noindent{\it Proof.}\ }
% \def\ep{\hfill $\Box$}


\begin{document}

%itemshape
\setbeamertemplate{itemize item}{\scriptsize\raise1.25pt\hbox{\donotcoloroutermaths$\bullet$}}
\setbeamertemplate{itemize subitem}{\tiny\raise1.5pt\hbox{\donotcoloroutermaths$\circ$}}
\setbeamertemplate{itemize subsubitem}{\tiny\raise1.5pt\hbox{\donotcoloroutermaths$\blacktriangleright$}}
%default value for spacing
\plitemsep 0.1in
\pltopsep 0.03in
\setlength{\parskip}{0.15in}
%\setlength{\parindent}{-0.5in}
\setlength{\abovedisplayskip}{0.07in}
\setlength{\mathindent}{0cm}
\setbeamertemplate{frametitle continuation}{[\insertcontinuationcount]}

\setlength{\leftmargini}{0.5cm}
\setlength{\leftmarginii}{0.5cm}

\setlength{\fboxrule}{0.05pt}
\setlength{\fboxsep}{5pt}

\begin{frame}
  \titlepage
\end{frame}

% % Uncomment these lines for an automatically generated outline.
% \begin{frame}{Outline}
% % \tableofcontents
% \plitemsep 0.1in
% \bci
% \item

% \item
% \eci
% \end{frame}

% START START START START START START START START START START START START START

%%%%%%%%%%%%%%%%%%%%%%%%%%%%%%%%%%%%%%%%%%%%%%%%%%%%%%
\begin{frame}{Roadmap}

\plitemsep 0.1in

\bce[(1)]


\item Derived distribution of $Y=g(X)$ or $Z=g(X,Y)$

\item Derived distribution of $Z=X+Y$

\item Covariance: Degree of dependence between two rvs.

\item Correlation coefficient

\item Conditional expectation and law of iterative expectations

\item Conditional variance and law of total variance

\item Random number of sum of random variables

\ece



\end{frame}

%%%%%%%%%%%%%%%%%%%%%%%%%%%%%%%%%%%%%%%%%%%%%%%%%%%%%%
\section{L5(1)}
\begin{frame}{Roadmap}


\plitemsep 0.1in

\bce[(1)]

\item \redf{Derived distribution of $Y=g(X)$ or $Z=g(X,Y)$}

\item \grayf{Derived distribution of $Z=X+Y$

\item Covariance: Degree of dependence between two rvs.


\item Correlation coefficient

\item Conditional expectation and law of iterative expectations

\item Conditional variance and law of total variance

\item Random number of sum of random variables}


\ece

\end{frame}

%%%%%%%%%%%%%%%%%%%%%%%%%%%%%%%%%%%%%%%%%%%%%%%%%%%%%%
\begin{frame}{Derived Distribution: $Y=g(X)$}

\plitemsep 0.1in

\bci
\item<1-> Given the PDF of $X,$ What is the PDF of $Y=g(X)$?

\item<2-> Wait! Didn't we cover this topic? No. We covered just $\expect{g(X)}.$

\item<3-> Examples: $Y=X,$ $Y=X+1,$ $Y=X^2,$ etc.

\item<4-> What are easy or difficult cases?

\item<5-> Easy cases

\bci
\item Discrete

\item Linear: $Y=aX +b$
\eci
\eci


\end{frame}

%%%%%%%%%%%%%%%%%%%%%%%%%%%%%%%%%%%%%%%%%%%%%%%%%%%%%%
\begin{frame}{Discrete Case}

\plitemsep 0.1in

\bci
\item<1-> Take all values of $x$ such that $g(x) = y,$ i.e.,
\aleq{
\py &= \cprob{g(X) = y} \cr
&= \sum_{x:g(x)=y} \px
}

\item[]
\mytwocols{0.4}
{
\aleq{
\onslide<3->{
p_Y(3) &= p_X(2) + p_X(3) = 0.1 + 0.2 = 0.3\cr
p_Y(4) &= p_X(4) + p_X(5) = 0.3+0.4 = 0.7
}
}
}
{
\centering
\onslide<2->{\mypic{0.8}{L5_derived_disc.png}}
}
\eci


\end{frame}

%%%%%%%%%%%%%%%%%%%%%%%%%%%%%%%%%%%%%%%%%%%%%%%%%%%%%%
\begin{frame}{Linear: $Y = aX +b,$ $a \neq 0$, $X$: Continuous}

\plitemsep 0.1in

\bci []
\item
\aleq{
\onslide<1->{\text{If $a > 0,$}}&
\onslide<2->{\quad \Fy = \cprob{aX+b \leq y} = \cprob{X \le \frac{y-b}{a}} = F_X(\frac{y-b}{a})}\cr
& \onslide<3->{\rightarrow \fy = \frac{1}{a} f_X \left (\frac{y-b}{a} \right )}\cr
\onslide<1->{\text{If $a <0,$}}&
\onslide<4->{\quad \Fy = \cprob{aX+b \leq y} = \cprob{X \ge \frac{y-b}{a}}
 = 1- F_X(\frac{y-b}{a})}\cr
& \onslide<5->{\rightarrow \fy = -\frac{1}{a} f_X \left (\frac{y-b}{a} \right )}\cr
}
\vspace{-0.5in}
\item<6-> Therefore,
\mycolorbox{
$
\displaystyle \fy = \frac{1}{|a|} f_X\left(\frac{y-b}{a}\right)
$
}
\eci


\end{frame}

%%%%%%%%%%%%%%%%%%%%%%%%%%%%%%%%%%%%%%%%%%%%%%%%%%%%%%
\begin{frame}{Linear: $Y = aX +b,$ when $X$ is exponential}

\plitemsep 0.2in

\bci[]
\item
\aleq
{
\fx = \begin{cases}
\lambda \elambdax, & \text{if} \quad x \ge 0 \cr
0, & \text{otherwise}
\end{cases}
}

\item
\aleq
{
\fy= \begin{cases}
\frac{\lambda}{|a|}e^{-\lambda(y-b)/a}, & \text{if} \quad (y-b)/a \ge 0 \cr
0, & \text{otherwise}
\end{cases}
}

\item[$\bullet$] If $b = 0$ and $a>0,$ $Y$ is exponential with parameter $\frac{\lambda}{a}$, but generally not.

\eci


\end{frame}


%%%%%%%%%%%%%%%%%%%%%%%%%%%%%%%%%%%%%%%%%%%%%%%%%%%%%%
\begin{frame}{Linear: $Y = aX +b,$ when $X$ is normal}

\plitemsep 0.1in

\bci
\item Remember? Linear transformation preserves normality. Time to prove.

\mycolorbox
{
If $X \sim  \set{N}(\mu, \sigma^2) $, then for $a \neq 0$ and $b,$ $Y = aX +b \sim \set{N}(a\mu +b,a^2 \sigma^2).$
}

\item<2-> \proff
\aleq{
\fx = \frac{1}{\sqrt{2\pi}} e^{-(x-\mu)^2/2 \sigma^2}
}
\aleq{
\onslide<3->{\fy &= \frac{1}{|a|} f_X\left(\frac{y-b}{a}\right) = \frac{1}{|a|}\frac{1}{\sqrt{2\pi}}
\exp\left \{ -\left(\frac{y-b}{a}-\mu\right)^2/2 \sigma^2 \right \} \cr
&= \frac{1}{\sqrt{2\pi}|a| \sigma} \exp\left \{ - \frac{(y-b-a\mu)^2}{2 a^2 \sigma^2}   \right \}}
}
\eci


\end{frame}


%%%%%%%%%%%%%%%%%%%%%%%%%%%%%%%%%%%%%%%%%%%%%%%%%%%%%%
\begin{frame}{Generally, $Y=g(X)$, $X$: Continuous}

\mytwocols{0.75}
{

\medskip
\plitemsep 0.2in

\bce
\item<2->[Step 1.] Find the CDF of $Y$: $\Fy = \cprob{Y\leq y} = \cprob{g(X) \leq y}$
\item<2->[Step 2.] Differentiate: $\fy = \frac{dF_Y}{dy}(y)$
\ece

\medskip

\small

\onslide<3->{\redf{Ex1.} $Y = X^2.$
\aleq{
\Fy &= \cprob{X^2 \le y} = \cprob{-\sqrt{y} \le X \le \sqrt{y}} \cr
& = F_X(\sqrt{y}) - F_X(-\sqrt{y})\cr
\fy &= \frac{1}{2\sqrt{y}} f_X(\sqrt{y}) +\cr
& \frac{1}{2\sqrt{y}} f_X(-\sqrt{y}), \quad y \ge 0
}
}


}
{
\small

\onslide<4->{\redf{Ex2.} $X \sim \set{U}[0,1].$ $Y = \sqrt{X}.$
\aleq{
\Fy &= \cprob{\sqrt{X} \le y} = \cprob{X \le y^2} = y^2 \cr
\fy &= 2y, \quad 0\le y \le 1
}
}

\onslide<5->{
\redf{Ex3.} $X \sim \set{U}[0,2].$ $Y = X^3.$
\aleq{
\Fy &= \cprob{X^3 \le y} = \cprob{X \le \sqrt[3]{y}} = \frac{1}{2} y^{1/3}\cr
\fy &= \frac{1}{6}y^{-2/3}, \quad 0\le y \le 8
}
}

\onslide<6->{When $Y=g(X)$ is monotonic, a \redf{general formula} can be drawn (see the textbook at pp 207)}

}
\end{frame}

%%%%%%%%%%%%%%%%%%%%%%%%%%%%%%%%%%%%%%%%%%%%%%%%%%%%%%
\begin{frame}{Functions of multiple rvs: $Z=g(X,Y)$ (1)}

\onslide<1->{\noindent Basically, follow two-step approach: (i) CDF and (ii) differentiate. }

\medskip

\small
\onslide<2->{\redf{Ex1.} $X,Y \sim \set{U}[0,1],$ and $X \indep Y.$ $Z = \max(X,Y).$

\medskip
* $\cprob{X \le z} = \cprob{Y \le z} = z,\ z \in [0,1].$
}
\aleq{
\onslide<3->{
F_Z(z) &= \cprob{\max(X,Y) \le z} = \cprob{X \le z, Y \le z}\cr
&= \cprob{X \le z} \cprob{Y \le z} = z^2 \qquad \qquad \text{(from $X \indep Y$)}}\cr
& \cr
\onslide<4->{
\fz& = \begin{cases}
2z, & \text{if $0 \le z \le 1$}\cr
0, & \text{otherwise}
\end{cases}
}
}

\end{frame}


%%%%%%%%%%%%%%%%%%%%%%%%%%%%%%%%%%%%%%%%%%%%%%%%%%%%%%
\begin{frame}{Functions of multiple rvs: $Z=g(X,Y)$ (2)}

\noindent Basically, follow two step approach: (i) CDF and (ii) differentiate.

\medskip

\mytwocols{0.6}
{
\small

\onslide<2->{
\redf{Ex2.}
$X,Y \sim \set{U}[0,1],$ and $X \indep Y.$
$Z = Y/X.$}

\hfill \only<2>{\lecturemark{VIDEO PAUSE}}



\aleq{
\onslide<3->{\Fz &= \cprob{Y/X \le z}} \cr
\onslide<5->{ &=
 \begin{cases}
 z/2, & 0 \le z \le 1\cr
 1-1/2z, & z >1 \cr
 0,& \text{otherwise}
 \end{cases}
 }
}
\vspace{-0.5cm}
\aleq{
\onslide<6->{
\fz &=
 \begin{cases}
 1/2, & 0 \le z \le 1\cr
 1/(2z^2), & z >1 \cr
 0,& \text{otherwise}
 \end{cases}
}
}
}
{

\small

\onslide<4->{
- Depending on the value of $z,$ two cases need to be considered separately.

%\medskip
\mypic{0.95}{L5_pdf_tworvs.png}

%- Where do we use $X \indep Y$?
}

\medskip
\onslide<7->{\redf{(Note)} Sometimes, the problem is tricky, which requires careful case-by-case handing. :-)}

}
\end{frame}

%%%%%%%%%%%%%%%%%%%%%%%%%%%%%%%%%%%%%%%%%%%%%%%%%%%%%%
\section{L5(2)}
\begin{frame}{Roadmap}


\plitemsep 0.1in

\bce[(1)]

\item \grayf{Derived distribution of $Y=g(X)$ or $Z=g(X,Y)$}

\item \redf{Derived distribution of $Z=X+Y$}

\item \grayf{Covariance: Degree of dependence between two rvs.

\item Correlation coefficient

\item Conditional expectation and law of iterative expectations

\item Conditional variance and law of total variance

\item Random number of sum of random variables}


\ece

\end{frame}


%%%%%%%%%%%%%%%%%%%%%%%%%%%%%%%%%%%%%%%%%%%%%%%%%%%%%%
\begin{frame}{Functions of multiple rvs: $Z=X+Y,$ $X\indep Y$ (1)}

\plitemsep 0.1in
\bci
\item Sum of two independent rvs
\item<2-> A very basic case with many applications
\item<2-> Assume that $X,Y \in \integer$
\aleq{
\onslide<3->{
\orangef{\pz} &= \cprob{X+Y = z} = \sum_{\{(x,y): x+y=z \}}\cprob{X=x,Y=y}= \sum_{x}\cprob{X=x,Y=z-x}\cr
 &= \sum_{x}\cprob{X=x} \cprob{Y=z-x} = \orangef{\sum_{x}\px p_Y(z-x)}
}
}
\vspace{-0.3cm}
\item<4-> $\pz$ is called \redblank{5}{convolution} of the PMFs of $X$ and $Y.$
\eci

\end{frame}


%%%%%%%%%%%%%%%%%%%%%%%%%%%%%%%%%%%%%%%%%%%%%%%%%%%%%%
\begin{frame}{Functions of multiple rvs: $Z=X+Y,$ $X\indep Y$ (2)}

\mytwocols{0.6}
{
\small
\plitemsep 0.15in
\bci
\item Convolution: $\pz = \sum_{x} \px p_Y(z-x)$
\item Interpretation for a given $z$:
\bce[(i)]
\item Flip (horizontally) the PMF of Y ($p_{Y}(-x)$)
\item Put it underneath the PMF of X
\item Right-shift the flipped PMF by $z$ ($p_{Y}(-x+z)$)
\ece
\eci
}
{
\exam $z=3$
\mypic{0.95}{L5_conv_ex.png}
}

\end{frame}


%%%%%%%%%%%%%%%%%%%%%%%%%%%%%%%%%%%%%%%%%%%%%%%%%%%%%%
\begin{frame}{$Y=X+Y,$ $X\indep Y$: Continuous}

\mytwocols{0.7}
{

\plitemsep 0.07in
\bci

\item<1-> Same logic as the discrete case

\aleq{
\fz = \int_{-\infty}^\infty \fx f_{Y}(z-x) dx
}



\item<3-> Youtube animation for convolution:
\url{https://www.youtube.com/watch?v=C1N55M1VD2o}

\eci
}
{
For a fixed $z,$

\onslide<2->{\mypic{0.95}{L5_convolution_cont_ex.png}}
}
\end{frame}


%%%%%%%%%%%%%%%%%%%%%%%%%%%%%%%%%%%%%%%%%%%%%%%%%%%%%%
\begin{frame}{Example}

\plitemsep 0.1in
\bci

\item \exam $X,Y \sim \set{U}[0,1]$ and $X \indep Y.$ What is the PDF of $Z = X+Y$?
Draw the PDF of $Z$.

\onslide<2->{\mypic{0.5}{L5_convolution_ex.png}}

\eci
\end{frame}

%%%%%%%%%%%%%%%%%%%%%%%%%%%%%%%%%%%%%%%%%%%%%%%%%%%%%%
\begin{frame}{Convolution in Image Processing}

\vspace{2cm}

{\large \url{https://www.youtube.com/watch?v=MQm6ZP1F6ms}}
\end{frame}


%%%%%%%%%%%%%%%%%%%%%%%%%%%%%%%%%%%%%%%%%%%%%%%%%%%%%%
\begin{frame}{$Y=X+Y,$ $X\indep Y,$ Normal (1)}


\plitemsep 0.1in
\bci

\item<1-> Very special, but useful case

\bci
\item $X$ and $Y$ are \bluef{normal.}
\eci

\bigskip
\onslide<3->{
\myblock{Sum of two independent normal rvs}
{
$X \sim \set{N}(\mu_x, \sigma_x^2)$ and  $Y \sim \set{N}(\mu_x, \sigma_x^2)$
Then, $X+Y \sim \set{N}(\mu_x + \mu_y, \sigma_x^2 + \sigma_y^2)$
}
}

\item<4->  Why normal rvs are used to model the \orangef{sum of random noises}.

\item<4-> \redf{Extension.} The sum of \redf{finitely many} independent normals is also normal.
\eci

\end{frame}

%%%%%%%%%%%%%%%%%%%%%%%%%%%%%%%%%%%%%%%%%%%%%%%%%%%%%%
\begin{frame}{$Y=X+Y,$ $X\indep Y$, Normal (2)}

\plitemsep 0.2in
\bci[]
\item
\aleq
{
\fz &= \int_{-\infty}^\infty \fx f_{Y}(z-x) dx \cr
&=\int_{-\infty}^\infty \frac{1}{\sqrt{2\pi} \sigma_x} \exp\left\{ - \frac{(x-\mu_x)^2}{2 \sigma_x^2} \right\} \frac{1}{\sqrt{2\pi} \sigma_y} \exp\left\{ - \frac{(z-x-\mu_y)^2}{2 \sigma_y^2} \right\} dx
}
\item[$\bullet$] The details of integration is a little bit tedious. :-)
\aleq{
\fz = \frac{1}{\sqrt{2\pi(\sigma_x^2 + \sigma_y^2)} } \exp\left\{ - \frac{(z-\mu_x-\mu_y)^2}{2(\sigma_x^2 + \sigma_y^2)} \right\}
}
\eci
\end{frame}

%%%%%%%%%%%%%%%%%%%%%%%%%%%%%%%%%%%%%%%%%%%%%%%%%%%%%%
\section{L5(3)}
\begin{frame}{Roadmap}


\plitemsep 0.1in

\bce[(1)]

\item \grayf{Derived distribution of $Y=g(X)$ or $Z=g(X,Y)$}

\item \grayf{Derived distribution of $Z=X+Y$}

\item \redf{Covariance: Degree of dependence between two rvs}

\item \grayf{Correlation coefficient

\item Conditional expectation and law of iterative expectations

\item Conditional variance and law of total variance

\item Random number of sum of random variables}


\ece

\end{frame}




% %%%%%%%%%%%%%%%%%%%%%%%%%%%%%%%%%%%%%%%%%%%%%%%%%%%%%%
% \begin{frame}{Dependence Degree: Motivating Example}
%
% \plitemsep 0.1in
%
% \bci
%
% \item covariance의 필요성을 이야기해주는 example을 찾아서 먼저 이야기를 해준다.
%
% \eci
%
% \end{frame}


%%%%%%%%%%%%%%%%%%%%%%%%%%%%%%%%%%%%%%%%%%%%%%%%%%%%%%
\begin{frame}{Making a Metric of Dependence Degree}

\plitemsep 0.05in

\bci

\item<1-> Goal: Given two rvs $X$ and $Y$, assign some number that quantifies the degree of their dependence.
\bci
\item<2-> feeling/weather, university ranking/annual salary,
\eci


\item<3-> Requirements
\bce[\bf R1.]
\item<4-> Increases (resp. decreases) as they become more (resp. less) dependent. 0 when they are independent.
\item<5-> Shows the `direction' of dependence by + and -
\item<6-> Always bounded by some numbers (i.e., dimensionless metric). For example, $[-1,1]$
\ece

\item<7-> Good engineers: Good at making good metrics

\smallskip
\bci
\item Metric of how our society is economically polarized
%\item A lot of metrics in our professional sports leagues (baseball, basketball, etc)
\item Cybermetrics in MLB (Major League Baseball): \url{http://m.mlb.com/glossary/advanced-stats}
\eci

\eci

\end{frame}

%%%%%%%%%%%%%%%%%%%%%%%%%%%%%%%%%%%%%%%%%%%%%%%%%%%%%%
\begin{frame}{OK. Let's Design!}

\plitemsep 0.04in

\bci

\item<1-> Simple case: $\expect{X} =\mu_x = 0$ and $\expect{Y}=\mu_Y = 0$
\item<2-> Dependent: Positive (If $X \uparrow$, $Y \uparrow$) or Negative (If $X \uparrow,$ $Y \downarrow$)

\item<3-> What about $\expect{XY}$? Seems good.

\bci
\item<4-> $\expect{XY} = \expect{X}\expect{Y}=0$ when $X \indep Y$
\item<5-> More data points (thus increases) when $xy >0$ (both positive or negative)
\item<6-> $|\expect{XY}|$ also quantifies the \orangef{amount of spread}.
\eci

\mytwocols{0.35}
{
\vspace{-0.3cm}
\onslide<5->{\mypic{0.8}{L5_cov_ex.png}}
}
{
\bigskip
\onslide<6->{\redf{(Q)} What about $\expect{X+Y}$?}
\bci
\item<7-> When they are positively dependent, but have negative values?
\eci
}

\eci

\end{frame}

%%%%%%%%%%%%%%%%%%%%%%%%%%%%%%%%%%%%%%%%%%%%%%%%%%%%%%
\begin{frame}{What If $\mu_X \neq 0, \mu_Y \neq 0$?}

\plitemsep 0.1in

\bci

\item<2-> \redf{Solution:} Centering. $X \rightarrow X - \mu_X$ and $Y \rightarrow Y-\mu_Y$
\onslide<3->{
\myblock{Covariance}
{
$\cov{X,Y} = \bexpect{(X- \expect{X})\cdot (Y-\expect{Y})}$
}}

\item<4-> After some algebra, $\cov{X,Y} = \expect{XY} - \expect{X}\expect{Y}$

\item<5-> $X \indep Y$ $\imp$ $\cov{X,Y}=0$

\item<6-> $\cov{X,Y}=0$ $\imp$ $X \indep Y$? NO.

\item<7-> When $\cov{X,Y}=0,$ we say that $X$ and $Y$ are \redblank{8}{uncorrelated.}

\eci

\end{frame}

%%%%%%%%%%%%%%%%%%%%%%%%%%%%%%%%%%%%%%%%%%%%%%%%%%%%%%
\begin{frame}{Example: $\cov{X,Y}=0,$ but not independent}

\plitemsep 0.1in

\bci

\item $p_{X,Y}(1,0) = p_{X,Y}(0,1) = p_{X,Y}(-1,0) = p_{X,Y}(0,-1) = 1/4.$

\item<2-> $\expect{X} = \expect{Y}=0,$ and $\expect{XY}=0.$ So, $\cov{X,Y}=0$

\item<3-> Are they independent? No, because if $X=1$, then we should have $Y=0.$

\eci
\centering
\mypic{0.3}{L5_cov_notind.png}
\end{frame}

%%%%%%%%%%%%%%%%%%%%%%%%%%%%%%%%%%%%%%%%%%%%%%%%%%%%%%
\begin{frame}{Some Properties}

\plitemsep 0.07in

\vspace{-0.3in}
\bci []

\item<2-> \aleq{
\cov{X,X}=\cvar{X}
}

\item<3-> \aleq{
\cov{aX +b, Y} &= \expect{(aX+b)Y} - \expect{aX+b}\expect{Y} = a\cdot \cov{X,Y}
}

\item<4->  \aleq{
\cov{X,Y+Z} &= \expect{X(Y+Z)} - \expect{X}\expect{Y+Z}= \cov{X,Y} + \cov{X,Z}
}

\item<5-> \aleq{
\var{X+Y} &= \expect{(X+Y)^2} - (\expect{X+Y})^2 = \var{X} + \var{Y} + 2\cov{X,Y}
}

\item<6-> \aleq{
\bvar{\sum X_i} &= \sum {\var{X_i}} + \sum_{i\neq j} \cov{X_i,X_j}
}

\eci

\end{frame}


%%%%%%%%%%%%%%%%%%%%%%%%%%%%%%%%%%%%%%%%%%%%%%%%%%%%%%
\begin{frame}{Example: The hat problem in Lecture 3. Remember?}

\bigskip

\mytwocols{0.7}
{
\small
\plitemsep 0.05in
\bci

\item $n$ people throw their hats in a box and then pick one at random

\item $X$: number of people with their own hat

\item \redf{(Q)} $\var{X}$

\item \bluef{Key step 1.} Define a rv $X_i=1$ if $i$ selects own hat and $0$ otherwise. Then, $X = \sum_{i=1}^n X_i.$

\item \bluef{Key step 2.} Are $X_i$s are independent?

\medskip
\item<2-> $X_i \sim \text{Bern}(1/n).$ Thus, $\expect{X_i} = 1/n$ and $\var{X_i} = \frac{1}{n}(1-\frac{1}{n})$

\eci
}
{
\small
%\sqeq
\plitemsep 0.01in
\bci [$\circ$]
\item \onslide<4->{For $i \neq j,$}
\setlength{\jot}{0pt}
\aleq{
\onslide<4->{\cov{X_i,X_j} &= \expect{X_i X_j} - \expect{X_i} \expect{X_j}\cr
&=\cprob{X_i =1 \text{ and } X_j=1} - \frac{1}{n^2} \cr
&=\cprob{X_i=1} \cprob{X_j=1 | X_i=1} - \frac{1}{n^2} \cr
&=\frac{1}{n} \frac{1}{n-1} - \frac{1}{n^2} = \frac{1}{n^2(n-1)}}
}
\vspace{-0.2in}
\aleq{
\onslide<3->{\var{X}& = \bvar{\sum X_i}\cr
&= \sum {\var{X_i}} + \sum_{i\neq j} \cov{X_i,X_j}}\cr
\onslide<5->{&= n \frac{1}{n}(1-\frac{1}{n}) + n(n-1)\frac{1}{n^2(n-1)}=1}\cr
}
\vspace{-0.1in}

\eci

}
\end{frame}

%%%%%%%%%%%%%%%%%%%%%%%%%%%%%%%%%%%%%%%%%%%%%%%%%%%%%%
\section{L5(4)}
\begin{frame}{Roadmap}


\plitemsep 0.1in

\bce[(1)]

\item \grayf{Derived distribution of $Y=g(X)$ or $Z=g(X,Y)$}

\item \grayf{Derived distribution of $Z=X+Y$}

\item \grayf{Covariance: Degree of dependence between two rvs}

\item \redf{Correlation coefficient}

\item \grayf{Conditional expectation and law of iterative expectations

\item Conditional variance and law of total variance

\item Random number of sum of random variables}


\ece

\end{frame}


%%%%%%%%%%%%%%%%%%%%%%%%%%%%%%%%%%%%%%%%%%%%%%%%%%%%%%
\begin{frame}{Bounding the metric: Correlation Coefficient}

\plitemsep 0.05in

\bci

\item<1-> Reqs. \bluef{\bf R1} and \bluef{\bf R2}  are satisfied.
%
\bce
\item<2->[\bf R3.] \magenf{Always bounded by some numbers (dimensionless metric)}
\ece

\item<3-> How? \redblank{4}{Normalization,} but by what?

\onslide<5->{
\myblock{Correlation Coefficient}
{
\aleq{
\rho(X,Y) = \bbexpect{\frac{(X - \mu_X)}{\redblank{6}{$\sigma_X$}} \cdot \frac{(Y-\mu_Y)}{\redblank{6}{$\sigma_Y$}}} = \frac{\cov{X,Y}}{\sqrt{\var{X}\var{Y}}}
}
}
}
\item<7-> \thm
\bce
\item<7-> $-1 \le \rho \le 1$ (proof at the next slide)
\item<8-> $|\rho|=1$ $\Leftrightarrow$ $X-\mu_X = c(Y-\mu_Y)$ for some constant $c$ ($c>0$ when $\rho=1$ and $c<0$ when $\rho=-1$). In other words, linear relation, meaning VERY related.
\ece
\eci

\end{frame}

% %%%%%%%%%%%%%%%%%%%%%%%%%%%%%%%%%%%%%%%%%%%%%%%%%%%%%%
% \begin{frame}{$-1 \le \rho \le 1$}
%
% \plitemsep 0.1in
%
%
% \bci
%
% \item \magenf{Schwarz inequaility.} For any rvs $X$ and $Y,$ \bluef{$ \big(\cexpect{XY} \big)^2 \le \cexpect{X^2} \cexpect{Y^2}$}
% \item<2-> Proof of $-1 \le \rho \le 1$:
%
% {\small
% Let $\tilde{X} = X - \cexpect{X}$ and $\tilde{Y} = Y - \cexpect{Y}.$ Then,
% $\displaystyle
% \big(\rho(X,Y) \big)^2 = \frac{\big(\expect{\tilde{X}\tilde{Y}} \big)^2}{\cexpect{\tilde{X}^2} \cexpect{\tilde{Y}^2}} \le 1
% $
% }
% \item<3-> Proof of SI: Assume $\cexpect{Y^2} \neq 0.$ Otherwise, the result immediately follows.
% \aleq{
% 0 &\le \bbexpect{\left(X - \frac{\cexpect{XY}}{\cexpect{Y^2}}Y\right)^2} =
% \bbexpect{X^2 -2 \frac{\cexpect{XY}}{\cexpect{Y^2}} XY + \frac{(\expect{XY})^2}{(\expect{Y^2})^2}Y^2 }\cr
% &= \cexpect{X^2} -2 \frac{\cexpect{XY}}{\cexpect{Y^2}} \cexpect{XY} + \frac{(\expect{XY})^2}{(\expect{Y^2})^2}\cexpect{Y^2} = \cexpect{X^2} - \frac{(\expect{XY})^2}{\cexpect{Y^2}}
% }
% \eci
%
% \end{frame}

%%%%%%%%%%%%%%%%%%%%%%%%%%%%%%%%%%%%%%%%%%%%%%%%%%%%%%
\begin{frame}{1. $-1 \le \rho \le 1$}

\plitemsep 0.1in


\bci

\item \magenf{Cauchy-Schwarz inequaility.} For any rvs $X$ and $Y,$ \bluef{$ \big(\cexpect{XY} \big)^2 \le \cexpect{X^2} \cexpect{Y^2}$}
\item<2-> \redf{Proof of $-1 \le \rho \le 1$:}

{\small
Let $\tilde{X} = X - \cexpect{X}$ and $\tilde{Y} = Y - \cexpect{Y}.$ Then,
$\displaystyle
\big(\rho(X,Y) \big)^2 = \frac{\big(\expect{\tilde{X}\tilde{Y}} \big)^2}{\cexpect{\tilde{X}^2} \cexpect{\tilde{Y}^2}} \le 1
$
}
\item<3-> \redf{Proof of CSI:} For any constant $a,$
\aleq{
0 &\le \bbexpect{\left(X - a Y\right)^2} =
\bbexpect{X^2 -2 a XY + a^2Y^2 }= \cexpect{X^2} -2 a \cexpect{XY} + a^2\cexpect{Y^2}
}
\onslide<4->{
Now, choose $a = \frac{\cexpect{XY}}{\cexpect{Y^2}}.$ Then,
\aleq{
\cexpect{X^2} -2 \frac{\cexpect{XY}}{\cexpect{Y^2}} \cexpect{XY} + \frac{(\expect{XY})^2}{(\expect{Y^2})^2}\cexpect{Y^2} = \cexpect{X^2} - \frac{(\expect{XY})^2}{\cexpect{Y^2}} \ge 0
}}
\eci


\end{frame}

%%%%%%%%%%%%%%%%%%%%%%%%%%%%%%%%%%%%%%%%%%%%%%%%%%%%%%
\begin{frame}{2. $|\rho|=1$ $\Leftrightarrow$ $X-\mu_X = c(Y-\mu_Y)$}

\newcommand{\tX}{\tilde{X}}
\newcommand{\tY}{\tilde{Y}}

\plitemsep 0.1in

\bci
\item[($\Rightarrow$)] Suppose that $|\rho|=1$. In the proof of CSI,
\aleq{
\bbexpect{\left(\tX - \frac{\cexpect{\tX\tY}}{\cexpect{\tY^2}} \tY\right)^2} = \cexpect{\tX^2} - \frac{(\expect{\tX\tY})^2}{\cexpect{\tY^2}} = \cexpect{\tX^2}(1-\rho^2) = 0
}
\aleq{
\tX - \frac{\cexpect{\tX\tY}}{\cexpect{\tY^2}} Y = 0 \leftrightarrow \tX = \frac{\cexpect{\tX\tY}}{\cexpect{\tY^2}} \tY
= \rho \sqrt{\frac{\cexpect{\tX^2}}{\cexpect{\tY^2}}} \tY
}

\item[($\Leftarrow$)] If $\tY = c\tX,$ then
\aleq{
\rho(X,Y) = \frac{\cexpect{\tX c \tX}}{\sqrt{\expect{\tX^2} \expect{(c \tX)^2}}} = \frac{c}{|c|}
}


\eci

\end{frame}

%%%%%%%%%%%%%%%%%%%%%%%%%%%%%%%%%%%%%%%%%%%%%%%%%%%%%%
\section{L5(5)}
\begin{frame}{Roadmap}


\plitemsep 0.1in

\bce[(1)]

\item \grayf{Derived distribution of $Y=g(X)$ or $Z=g(X,Y)$}

\item \grayf{Derived distribution of $Z=X+Y$}

\item \grayf{Covariance: Degree of dependence between two rvs}

\item \grayf{Correlation coefficient}

\item \redf{Conditional expectation and law of iterative expectations}

\item \grayf{Conditional variance and law of total variance

\item Random number of sum of random variables}


\ece

\end{frame}



%%%%%%%%%%%%%%%%%%%%%%%%%%%%%%%%%%%%%%%%%%%%%%%%%%%%%%
\begin{frame}{A Special Random Variable}

\mytwocols{0.6}
{
\plitemsep 0.1in

\abovedisplayskip=5pt
\belowdisplayskip=5pt

\bci
\item<1-> Consider a rv $Y$, such that
\aleq{
Y &= \begin{cases}
0,& \text{w.p.  } 1/4 \cr
1,& \text{w.p.  } 1/4 \cr
2,& \text{w.p.  } 1/2
\end{cases}
}

\item<2-> If $h(y) = y^2,$ then a new rv $h(Y)$ is:
\aleq{
h(Y) &= \begin{cases}
0,& \text{w.p.  } 1/4 \cr
1,& \text{w.p.  } 1/4 \cr
4,& \text{w.p.  } 1/2
\end{cases}
}
\eci
}
{
\plitemsep 0.05in
\abovedisplayskip=5pt
\belowdisplayskip=5pt

\bci
\item<3-> Consider other rv $X$, which, we assume, has:
\aleq{
\onslide<4->{\redf{g(y)}= } \expect{X|Y=y}=\begin{cases}
3,& \text{if $y=0$}\cr
8,& \text{if $y=1$}\cr
9,& \text{if $y=2$}
\end{cases}
}

\item<5-> Then, a rv $\redf{g(Y)}$ is:
\aleq{
g(Y) &= \begin{cases}
3,& \text{w.p.  } 1/4 \cr
8,& \text{w.p.  } 1/4 \cr
9,& \text{w.p.  } 1/2
\end{cases}
}
\eci
}

\onslide<6->{- The rv $g(Y)$ looks special, so let's give a fancy notation to it.}

\onslide<7->{- What about? $X_{exp}(Y)$, $\expect{X_Y}$, $\expecti{X}{Y}$?}

\end{frame}

%%%%%%%%%%%%%%%%%%%%%%%%%%%%%%%%%%%%%%%%%%%%%%%%%%%%%%
\begin{frame}{Conditional Expectation $\expect{X|Y}$}

\plitemsep 0.1in

% \abovedisplayskip=5pt
% \belowdisplayskip=5pt

\myblock{Conditional Expectation}{
A random variable $g(Y) =$ \redblank{2}{$\expect{X|Y}$}, called \redblank{2}{conditional expectation of $X$ given $Y$}, takes the value $g(y) = \expect{X|Y=y},$ if $Y$ happens to take the value $y.$
}

\bci
\item<2-> A function of $Y$
\item<3-> A random variable
\item<4-> Thus, having a distribution, expectation, variance, all the things that a
random variable has.
\item<5-> Often confusing because of the notation.
\eci

\end{frame}

%%%%%%%%%%%%%%%%%%%%%%%%%%%%%%%%%%%%%%%%%%%%%%%%%%%%%%
\begin{frame}{Expectation of $\expect{X|Y}$}

\plitemsep 0.1in

% \abovedisplayskip=5pt
% \belowdisplayskip=5pt

\myblock{Expectation of Conditional Expectation}{
\aleq
{
\bexpect{\expect{X|Y}} = \expect{X}, \quad \text{Law of iterated expectations}
}
}

\redf{Proof.}
\aleq
{
\bexpect{\expect{X|Y}} &= \sum_{y} \expect{X | Y=y} \py \cr
&= \expect{X}
}
\end{frame}

%%%%%%%%%%%%%%%%%%%%%%%%%%%%%%%%%%%%%%%%%%%%%%%%%%%%%%
\begin{frame}{Examples and Meaning}

\mytwocols{0.6}
{
\plitemsep 0.1in

% \abovedisplayskip=5pt
% \belowdisplayskip=5pt

\bci
\item<2-> Stick of length $l$
\item<2-> Uniformly break at point $Y,$ and break what is left uniformly at point $X.$

\item<3-> $\expect{X | Y=y} = y/2$

\item<3-> $\expect{X | Y} = Y/2$

\item<4-> $\expect{X} = \expect{\expect{X|Y}} = \expect{Y/2}=\frac{1}{2}\frac{l}{2} = l/4$
\eci
}
{
\plitemsep 0.1in

% \abovedisplayskip=5pt
% \belowdisplayskip=5pt

\bci
\item<5-> Forecasts on sales: calculating expected value, given any available information

\item<6-> $X:$ February sales

\item<6-> Forecast in the beg. of the year: $\expect{X}$

\item<7-> End of Jan. new information $Y=y$ (Jan. sales)

Revised forecast: $\expect{X|Y=y}$

Revised forecast $\neq$ $\expect{X}$

\item<8-> Law of iterated expectations
$\expect{\text{revised forecast}} = \text{original one}$

\eci

}
\end{frame}

%%%%%%%%%%%%%%%%%%%%%%%%%%%%%%%%%%%%%%%%%%%%%%%%%%%%%%
\begin{frame}{Example: Averaging Quiz Scores by Section}

\mytwocols{0.7}
{
\small
\plitemsep 0.05in

% \abovedisplayskip=5pt
% \belowdisplayskip=5pt

\bci
\item A class: $n$ students, student $i$'s quiz score: $x_i$
\item<2-> Average quiz score: $m = \frac{1}{n} \sum_{i=1}^n x_i$
\item<3-> Students: partitioned into sections $A_1,$ \ldots, $A_k$ and $n_s$: number of students in section $s$
\item<4-> average score in section $s$ = $m_s = \frac{1}{n_s}\sum_{i \in A_s} x_i$

\medskip
\item<5-> whole average: (i) taking the average $m_s$ of each section and
(ii) forming a weighted average
\aleq{
\sum_{s=1}^k \frac{n_s}{n} m_s = \sum_{s=1}^k\frac{n_s}{n} \frac{1}{n_s}\sum_{i \in A_s} x_i = \frac{1}{n} \sum_{i=1}^n x_i = m
}
\eci
}
{
\plitemsep 0.05in

% \abovedisplayskip=5pt
% \belowdisplayskip=5pt
\small

\bci
\item<6-> Understanding from $\bexpect{\expect{X|Y}} = \expect{X}$

\item<7-> $X$: score of a randomly chosen student, $Y$: section of a student ($\in \{1, \ldots, k\}$)

\aleq{
m &= \cexpect{X} = \bexpect{\expect{X|Y}}\cr
&=\sum_{s=1}^k \cexpect{X | Y=s} \cprob{Y=s} \cr
&= \onslide<8->{\sum_{s=1}^k \left(\frac{1}{n_s} \sum_{i \in A_s} x_i \right)\frac{n_s}{n} = \sum_{s=1}^k m_s \frac{n_s}{n}}
}
\eci

}
\end{frame}

%%%%%%%%%%%%%%%%%%%%%%%%%%%%%%%%%%%%%%%%%%%%%%%%%%%%%%
\section{L5(6)}
\begin{frame}{Roadmap}


\plitemsep 0.1in

\bce[(1)]

\item \grayf{Derived distribution of $Y=g(X)$ or $Z=g(X,Y)$}

\item \grayf{Derived distribution of $Z=X+Y$}

\item \grayf{Covariance: Degree of dependence between two rvs}

\item \grayf{Correlation coefficient}

\item \grayf{Conditional expectation and law of iterative expectations}

\item \redf{Conditional variance and law of total variance}

\item \grayf{Random number of sum of random variables}


\ece

\end{frame}

%%%%%%%%%%%%%%%%%%%%%%%%%%%%%%%%%%%%%%%%%%%%%%%%%%%%%%
\begin{frame}{Conditional Variance $\var{X|Y}$}


% \abovedisplayskip=5pt
% \belowdisplayskip=5pt
\onslide<1->{$\var{X} = \expect{(X-\expect{X})^2}$}

\onslide<2->{$g(y) = \var{X\redf{|Y=y}} = \expect{(X-\expect{X\redf{|Y=y}})^2 | \redf{Y=y}}$}

\onslide<3->{$g(Y) = \var{X\redf{|Y}} = \expect{(X-\expect{X\redf{|Y}})^2 | \redf{Y}}$}

\onslide<4->{
\myblock{Conditional Variance}{
A random variable $g(Y) =$ \redblank{5}{$\var{X|Y}$} and called \redblank{5}{conditional variance of $X$ given $Y$}, takes the value $g(y) = \var{X|Y=y},$ if $Y$ happens to take the value $y.$
}}

\vspace{-0.3cm}
\plitemsep 0.03in

\bci
\item<6-> A function of $Y$
\item<6-> A random variable
\item<6-> Thus, having a distribution, expectation, variance, all the things that a
random variable has
\eci

\end{frame}

%%%%%%%%%%%%%%%%%%%%%%%%%%%%%%%%%%%%%%%%%%%%%%%%%%%%%%
\begin{frame}{Expectation and Variance of $\expect{X|Y}$ and $\var{X|Y}$}

\Large

\centering

\renewcommand{\arraystretch}{1.7}
\setlength{\tabcolsep}{12pt}
\begin{tabular}{@{}l|l|l@{}} \toprule
 & $\expect{X|Y}$ & $\var{X|Y}$ \\ \midrule \midrule
Expectation& $\bexpect{\cexpect{X|Y}}$& $\bexpect{\cvar{X|Y}}$ \\  \midrule
Variance& $\bvar{\cexpect{X|Y}}$& $\bvar{\cvar{X|Y}}$ \\ \bottomrule
\end{tabular}

\end{frame}

%%%%%%%%%%%%%%%%%%%%%%%%%%%%%%%%%%%%%%%%%%%%%%%%%%%%%%
\begin{frame}{Law of Total Variance}

\myblock{Law of total variance (LTV)}{
$$
\var{X} = \onslide<2->{\bexpect{\cvar{X|Y}} + \var{\cexpect{X|Y}}}
$$
}

\redf{Proof.}
\begin{align}
\onslide<3->{\cvar{X|Y} &= \expect{X^2 |Y} - (\expect{X|Y})^2} \nonumber \\
\onslide<4->{\bexpect{\cvar{X|Y}} &= \expect{X^2} - \bexpect{(\expect{X|Y})^2} \label{eq:1}}\\
\onslide<5->{\bvar{\cexpect{X|Y}} &= \bexpect{(\expect{X|Y})^2} - (\bexpect{\cexpect{X|Y}})^2 = \bexpect{(\expect{X|Y})^2} - (\expect{X})^2 \label{eq:2}}
\end{align}

\onslide<6->{\eqref{eq:1} + \eqref{eq:2} = $\expect{X^2}$ + $(\expect{X})^2$ = $\var{X}$}

\end{frame}

%%%%%%%%%%%%%%%%%%%%%%%%%%%%%%%%%%%%%%%%%%%%%%%%%%%%%%
\begin{frame}{Example: Averarging Quiz Scores by Section}

\plitemsep 0.07in

\bci
\item Same setting as that in page 36

\item $X$: score of a randomly chosen student, $Y$: section of a student ($\in \{1, \ldots, k\}$)

\item<2-> Let's intuitively understand: $\var{X} = \bexpect{\cvar{X|Y}} + \var{\cexpect{X|Y}}$

\item<3-> $\expect{\cvar{X|Y}} = \sum_{k=1}^s \cprob{Y=s}\cvar{X | Y=s} = \sum_{k=1}^s \frac{n_s}{n} \cvar{X | Y=s}$
\bci
\item Weighted average of the section variances
\item<5-> \redf{average score variability within individual sections}
\eci

\item<4-> $\var{\cexpect{X|Y}}$: variability of the average of the differenct sections
\bci
\item $\cexpect{X | Y=s}$: average score in section $s$
\item<5-> \redf{variability between sections}
\eci
\eci


\end{frame}

%%%%%%%%%%%%%%%%%%%%%%%%%%%%%%%%%%%%%%%%%%%%%%%%%%%%%%
\begin{frame}{Example: Stick-breaking}

\plitemsep 0.07in

\bci
\item Stick of length $l$
\item Uniformly break at point $Y,$ and break what is left uniformly at point $X.$
\item<2-> \question $\cvar{X}$?

\item<2-> LTV: $\var{X} = \bexpect{\cvar{X|Y}} + \var{\cexpect{X|Y}}$

\item<3-> \redf{Fact.} If a rv $X \sim \set{U}[0,\theta]$, then $\cvar{X} = \frac{\theta^2}{12}$

\item<4-> Since $X \sim \set{U}[0,Y],$ $\cvar{X | Y} = \frac{Y^2}{12}$ $\rightarrow$ $\expect{\var{X | Y}} = \frac{1}{12} \int_0^l \frac{1}{l} y^2 dy = \frac{l^2}{36}$

\item<5-> $\cexpect{X |Y} = Y/2$ $\rightarrow$ $\cvar{\expect{X |Y}} = \frac{1}{4} \var{Y} = \frac{1}{4} \frac{l^2}{12} = \frac{l^2}{48}$

\item<6-> $\cvar{X} = \frac{l^2}{36} + \frac{l^2}{48} = \frac{7l^2}{144}$
\eci


\end{frame}



%%%%%%%%%%%%%%%%%%%%%%%%%%%%%%%%%%%%%%%%%%%%%%%%%%%%%%
\section{L5(6)}
\begin{frame}{Roadmap}


\plitemsep 0.1in

\bce[(1)]

\item \grayf{Derived distribution of $Y=g(X)$ or $Z=g(X,Y)$}

\item \grayf{Derived distribution of $Z=X+Y$}

\item \grayf{Covariance: Degree of dependence between two rvs}

\item \grayf{Correlation coefficient}

\item \grayf{Conditional expectation and law of iterative expectations}

\item \grayf{Conditional variance and law of total variance}

\item \redf{Random number of sum of random variables}


\ece

\end{frame}


%%%%%%%%%%%%%%%%%%%%%%%%%%%%%%%%%%%%%%%%%%%%%%%%%%%%%%
\begin{frame}{Sum of a random number of rvs}

\plitemsep 0.1in

\bci
\item<1-> $N:$ number of stores visited (\redf{random})

\item<2-> $X_i$: money spent in store $i,$ independent of other $X_j$ and $N,$ $X_i$s are identically distributed with $\expect{X_i} = \mu$

\item<3-> $Y = X_1 + X_2 + \ldots X_N.$ What are $\expect{Y}$ and $\var{Y}$?

\item<4-> $\expect{Y} = \expect{\expect{Y|N}} = \expect{N\expect{X_i}} = \expect{N}\expect{X_i} = \mu \expect{N}$

\item<5-> $\var{Y} = \bexpect{\cvar{Y|N}} + \var{\cexpect{Y|N}}$  \onslide<9->{$=\expect{N}\var{X_i} + \mu^2 \var{N}$}
\aleq{
\onslide<6->{\cvar{\expect{Y|N}} &= \cvar{N\mu} = \mu^2 \var{N}} \cr
\onslide<7->{\var{Y|N} &= N \var{X_i}}\cr
\onslide<8->{\expect{\cvar{Y|N}} &= \expect{N \var{X_i}} = \expect{N}\var{X_i}}
}
% \mytwocols{0.4}
% {

% }
% {
% }

\eci


\end{frame}

\begin{frame}{}
\vspace{2cm}
\LARGE Questions?

\end{frame}

\begin{frame}{Review Questions}

\bce[1)]
\item What are the key steps to get the derived distributions of $Y=g(X)$ or $Z=g(X,Y)$?

\item How can we compute the distribution of $Z+X+Y$ when $X$ and $Y$ are independent?

\item What are covariance and correlation coefficient? Why do we need them?

\item Please explain the concepts of conditional expectation and conditional variance.

\ece

\end{frame}

\end{document}
