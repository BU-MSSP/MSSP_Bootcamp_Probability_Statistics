
\def\slidemode{
  \documentclass[fleqn,aspectratio=169]{beamer}
\usepackage{pgfpages}
}
\def\handoutmode{
  \documentclass[handout,fleqn,aspectratio=169]{beamer}
\usepackage{pgfpages}
\pgfpagesuselayout{resize to}[a4paper,landscape,border shrink=5mm]
}


%\def\pdfmode{handoutmode}
\def\pdfmode{slidemode}

\csname\pdfmode\endcsname


\usepackage[english]{babel}
%\usepackage[utf8x]{inputenc}
\usepackage{tikz}
\usepackage{courier}
\usepackage{array}
\usepackage{bold-extra}
%\usepackage{minted}
%\usepackage[thicklines]{cancel}
%\usepackage{fancyvrb}
\usepackage{kotex}
\usepackage{paralist}
\usepackage{collectbox}
\usepackage{amsmath}
\usepackage{mathtools}
\usepackage{nccmath}
\usepackage{bm}
\usepackage{booktabs}
\usepackage{tcolorbox}
\usepackage[absolute,overlay]{textpos}

\xdefinecolor{dianablue}{rgb}{0.18,0.24,0.31}
\xdefinecolor{darkblue}{rgb}{0.1,0.1,0.7}
\xdefinecolor{darkgreen}{rgb}{0,0.5,0}
\xdefinecolor{darkgrey}{rgb}{0.35,0.35,0.35}
\xdefinecolor{darkorange}{rgb}{0.8,0.5,0}
\xdefinecolor{darkred}{rgb}{0.7,0,0}
\definecolor{darkgreen}{rgb}{0,0.6,0}
\definecolor{mauve}{rgb}{0.58,0,0.82}


\title[]{Lecture 7: Law of Large Numbers and Central Limit Theorem}
\author{Yi, Yung (이융)}
\institute{EE210: Probability and Introductory Random Processes\\ KAIST EE}
\date{MONTH DAY, 2021}


%%%%%%%%%%%% real, integer notation
\newcommand{\real}{{\mathbb R}}
\newcommand{\integer}{{\mathbb Z}}

%%% set, vector, matrix
\newcommand{\set}[1]{\ensuremath{\mathcal #1}}
\renewcommand{\vec}[1]{\bm{#1}}
\newcommand{\mat}[1]{\bm{#1}}


%%% big parenthesis
\def\Bl{\Bigl}
\def\Br{\Bigr}
\def\lf{\left}
\def\ri{\right}


%%% floor notations
\newcommand{\lfl}{{\lfloor}}
\newcommand{\rfl}{{\rfloor}}
\newcommand{\floor}[1]{{\lfloor #1 \rfloor}}

%%% gradient
\newcommand{\grad}[1]{\nabla #1}

%%% definition
\newcommand{\eqdef}{\ensuremath{\triangleq}}
%%% imply
\newcommand{\imp}{\Longrightarrow}



\newcommand{\separator}{
%  \begin{center}
    \par\noindent\rule{\columnwidth}{0.3mm}
%  \end{center}
}

\newcommand{\mynote}[1]{{\it \color{red} [#1]}}







%%% equation alignment
\newcommand{\aleq}[1]{\begin{align*}#1\end{align*}}

%%%%%%%%%%%%%%%% colored emphasized font, blanked words

\newcommand{\empr}[1]{{\color{red}\emph{#1}}}
\newcommand{\empb}[1]{{\color{blue}\emph{#1}}}

%normal colored text
\newcommand{\redf}[1]{{\color{red} #1}}
\newcommand{\yellowf}[1]{{\color{yellow} #1}}
\newcommand{\bluef}[1]{{\color{blue} #1}}
\newcommand{\grayf}[1]{{\color{gray} #1}}
\newcommand{\magenf}[1]{{\color{magenta} #1}}
\newcommand{\greenf}[1]{{\color{green} #1}}
\newcommand{\cyanf}[1]{{\color{cyan} #1}}
\newcommand{\orangef}[1]{{\color{orange} #1}}


\newcommand{\blk}[1]{\underline{\mbox{\hspace{#1}}}}


\newcommand{\redblk}[1]{\framebox{\color{red} #1}}
\newcommand{\redblank}[2]{\framebox{\onslide<#1->{\color{red} #2}}}
\newcommand{\blueblk}[1]{\framebox{\color{blue} #1}}
\newcommand{\blueblank}[2]{\framebox{\onslide<#1->{\color{blue} #2}}}



\makeatletter
\newcommand{\mybox}{%
    \collectbox{%
        \setlength{\fboxsep}{1pt}%
        \fbox{\BOXCONTENT}%
    }%
}
\makeatother

\makeatletter
\newcommand{\lecturemark}{%
    \collectbox{%
        \setlength{\fboxsep}{1pt}%
        \fcolorbox{red}{yellow}{\BOXCONTENT}%
    }%
}
\makeatother

\usepackage{tcolorbox}
\newcommand{\mycolorbox}[1]{
\begin{tcolorbox}[colback=red!5!white,colframe=red!75!black]
#1
\end{tcolorbox}
}

%%%% figure inclusion
\newcommand{\mypic}[2]{
\begin{center}
\includegraphics[width=#1\textwidth]{#2}
\end{center}
}

\newcommand{\myinlinepic}[2]{
\makebox[0cm][r]{\raisebox{-4ex}{\includegraphics[height=#1]{#2}}}
}


%%%% itemized and enumerated list
\newcommand{\bci}{\begin{compactitem}}
\newcommand{\eci}{\end{compactitem}}
\newcommand{\bce}{\begin{compactenum}}
\newcommand{\ece}{\end{compactenum}}


%%%% making 0.5/0.5 two columns
%%%% how to use: first number: length of separation bar
% \mytwocols{0.6}
% {
% contents in the left column
% }
% {
% contents in the right column
% }
%%%%
\newcommand{\mytwocols}[3]{
\begin{columns}[T] \column{.499\textwidth} #2 \column{.001\textwidth} \rule{.3mm}{{#1}\textheight} \column{.499\textwidth} #3 \end{columns}}

\newcommand{\mythreecols}[4]{
\begin{columns}[T] \column{.31\textwidth} #2 \column{.001\textwidth} \rule{.3mm}{{#1}\textheight} \column{.31\textwidth} #3 \column{.001\textwidth} \rule{.3mm}{{#1}\textheight} \column{.31\textwidth} #4  \end{columns}}

\newcommand{\mysmalltwocols}[3]{
\begin{columns}[T] \column{.4\textwidth} #2 \column{.001\textwidth} \rule{.3mm}{{#1}\textheight} \column{.4\textwidth} #3 \end{columns}}

%%%% making two columns with customized ratios
%%%% how to use:
%first parameter: length of separation bar
%second parameter: ratio of left column
%third parameter: ratio of right column
% \mytwocols{0.6}{0.7}{0.29}
% {
% contents in the left column
% }
% {
% contents in the right column
% }
%%%%
\newcommand{\myvartwocols}[5]{
\begin{columns}[T] \column{#2\textwidth} {#4} \column{.01\textwidth} \rule{.3mm}{{#1}\textheight} \column{#3\textwidth} {#5} \end{columns}}

%%% making my block in beamer
%%% first parameter: title of block
%%% second parameter: contents of block
\newcommand{\myblock}[2]{
\begin{block}{#1} {#2}  \end{block}}

%%% independence notation
\newcommand{\indep}{\perp \!\!\! \perp}

%%%% probability with different shapes (parenthesis or bracket) and different sizes
%%% `i' enables us to insert the subscript to the probability
\newcommand{\bprob}[1]{\mathbb{P}\Bl[ #1 \Br]}
\newcommand{\prob}[1]{\mathbb{P}[ #1 ]}
\newcommand{\cbprob}[1]{\mathbb{P}\Bl( #1 \Br)}
\newcommand{\cprob}[1]{\mathbb{P}( #1 )}
\newcommand{\probi}[2]{\mathbb{P}_{#1}[ #2 ]}
\newcommand{\bprobi}[2]{\mathbb{P}_{#1}\Bl[ #2 \Br]}
\newcommand{\cprobi}[2]{\mathbb{P}_{#1}( #2 )}
\newcommand{\cbprobi}[2]{\mathbb{P}_{#1}\Bl( #2 \Br)}

%%%% expectation with different shapes (parenthesis or bracket) and different sizes
%%% `i' enables us to insert the subscript to the expectation
\newcommand{\expect}[1]{\mathbb{E}[ #1 ]}
\newcommand{\cexpect}[1]{\mathbb{E}( #1 )}
\newcommand{\bexpect}[1]{\mathbb{E}\Bl[ #1 \Br]}
\newcommand{\cbexpect}[1]{\mathbb{E}\Bl( #1 \Br)}
\newcommand{\bbexpect}[1]{\mathbb{E}\lf[ #1 \ri]}
\newcommand{\expecti}[2]{\mathbb{E}_{#1}[ #2 ]}
\newcommand{\bexpecti}[2]{\mathbb{E}_{#1}\Bl[ #2 \Br]}
\newcommand{\bbexpecti}[2]{\mathbb{E}_{#1}\lf[ #2 \ri]}

%%%% variance
\newcommand{\var}[1]{\text{var}[ #1 ]}
\newcommand{\bvar}[1]{\text{var}\Bl[ #1 \Br]}
\newcommand{\cvar}[1]{\text{var}( #1 )}
\newcommand{\cbvar}[1]{\text{var}\Bl( #1 \Br)}

%%%% covariance
\newcommand{\cov}[1]{\text{cov}( #1 )}
\newcommand{\bcov}[1]{\text{cov}\Bl( #1 \Br)}

%%% Popular pmf, pdf notation to avoid long typing
\newcommand{\px}{\ensuremath{p_X(x)}}
\newcommand{\py}{\ensuremath{p_Y(y)}}
\newcommand{\pz}{\ensuremath{p_Z(z)}}
\newcommand{\pxA}{\ensuremath{p_{X|A}(x)}}
\newcommand{\pyA}{\ensuremath{p_{Y|A}(y)}}
\newcommand{\pzA}{\ensuremath{p_{Z|A}(z)}}
\newcommand{\pxy}{\ensuremath{p_{X,Y}(x,y)}}
\newcommand{\pxcy}{\ensuremath{p_{X|Y}(x|y)}}
\newcommand{\pycx}{\ensuremath{p_{Y|X}(y|x)}}

\newcommand{\fx}{\ensuremath{f_X(x)}}
\newcommand{\Fx}{\ensuremath{F_X(x)}}
\newcommand{\fy}{\ensuremath{f_Y(y)}}
\newcommand{\Fy}{\ensuremath{F_Y(y)}}
\newcommand{\fz}{\ensuremath{f_Z(z)}}
\newcommand{\Fz}{\ensuremath{F_Z(z)}}
\newcommand{\fxA}{\ensuremath{f_{X|A}(x)}}
\newcommand{\fyA}{\ensuremath{f_{Y|A}(y)}}
\newcommand{\fzA}{\ensuremath{f_{Z|A}(z)}}
\newcommand{\fxy}{\ensuremath{f_{X,Y}(x,y)}}
\newcommand{\Fxy}{\ensuremath{F_{X,Y}(x,y)}}
\newcommand{\fxcy}{\ensuremath{f_{X|Y}(x|y)}}
\newcommand{\fycx}{\ensuremath{f_{Y|X}(y|x)}}

\newcommand{\fth}{\ensuremath{f_\Theta(\theta)}}
\newcommand{\fxcth}{\ensuremath{f_{X|\Theta}(x|\theta)}}
\newcommand{\fthcx}{\ensuremath{f_{\Theta|X}(\theta|x)}}

\newcommand{\pkcth}{\ensuremath{p_{X|\Theta}(k|\theta)}}
\newcommand{\fthck}{\ensuremath{f_{\Theta|X}(\theta|k)}}


%%%% indicator
\newcommand{\indi}[1]{\mathbf{1}_{ #1 }}

%%%% exponential rv.
\newcommand{\elambdax}{\ensuremath{e^{-\lambda x}}}

%%%% normal  rv.
\newcommand{\stdnormal}{\ensuremath{\frac{1}{\sqrt{2\pi}} e^{-x^2/2}}}
\newcommand{\gennormal}{\ensuremath{\frac{1}{\sigma\sqrt{2\pi}} e^{-(x-\mu)^2/2}}}

%%%%%% estimator, estimate
\newcommand{\hth}{\ensuremath{\hat{\theta}}}
\newcommand{\hTH}{\ensuremath{\hat{\Theta}}}
\newcommand{\MAP}{\ensuremath{\text{MAP}}}
\newcommand{\LMS}{\ensuremath{\text{LMS}}}
\newcommand{\LLMS}{\ensuremath{\text{L}}}
\newcommand{\ML}{\ensuremath{\text{ML}}}

%%%% colored text
\newcommand{\red}[1]{\color{red}#1}
\newcommand{\cyan}[1]{\color{cyan}#1}
\newcommand{\magenta}[1]{\color{magenta}#1}
\newcommand{\blue}[1]{\color{blue}#1}
\newcommand{\green}[1]{\color{green}#1}
\newcommand{\white}[1]{\color{white}#1}

\newcommand{\defi}{{\color{red} Definition.} }
\newcommand{\proff}{{\color{red} Proof.} }
\newcommand{\exam}{{\color{red} Example.} }
\newcommand{\question}{{\color{red} Question.} }
\newcommand{\thm}{{\color{red} Theorem.} }
\newcommand{\background}{{\color{red} Background.} }
\newcommand{\msg}{{\color{red} Message.} }

\newcommand{\bern}[1]{\ensuremath{\text{Bern}(#1)} }
\newcommand{\bp}[1]{\ensuremath{\text{BP}(#1)} }
\newcommand{\poisson}[1]{\ensuremath{\text{Poisson}(#1)} }
\newcommand{\pp}[1]{\ensuremath{\text{PP}(#1)} }


%%%%%%%%%%%%%%%%%%%%%%% old macros that you can ignore %%%%%%%%%%%%%%%%%%%%%%%%

% \def\un{\underline}
% \def\ov{\overline}


% \newcommand{\beq}{\begin{eqnarray*}}
% \newcommand{\eeq}{\end{eqnarray*}}
% \newcommand{\beqn}{\begin{eqnarray}}
% \newcommand{\eeqn}{\end{eqnarray}}
% \newcommand{\bemn}{\begin{multiline}}
% \newcommand{\eemn}{\end{multiline}}
% \newcommand{\beal}{\begin{align}}
% \newcommand{\eeal}{\end{align}}
% \newcommand{\beas}{\begin{align*}}
% \newcommand{\eeas}{\end{align*}}



% \newcommand{\bd}{\begin{displaymath}}
% \newcommand{\ed}{\end{displaymath}}
% \newcommand{\bee}{\begin{equation}}
% \newcommand{\eee}{\end{equation}}


% \newcommand{\vs}{\vspace{0.2in}}
% \newcommand{\hs}{\hspace{0.5in}}
% \newcommand{\el}{\end{flushleft}}
% \newcommand{\bl}{\begin{flushleft}}
% \newcommand{\bc}{\begin{center}}
% \newcommand{\ec}{\end{center}}
% \newcommand{\remove}[1]{}

% \newtheorem{theorem}{Theorem}
% \newtheorem{corollary}{Corollary}
% \newtheorem{prop}{Proposition}
% \newtheorem{lemma}{Lemma}
% \newtheorem{defi}{Definition}
% \newtheorem{assum}{Assumption}
% \newtheorem{example}{Example}
% \newtheorem{property}{Property}
% \newtheorem{remark}{Remark}

% \newcommand{\separator}{
%   \begin{center}
%     \rule{\columnwidth}{0.3mm}
%   \end{center}
% }

% \newenvironment{separation}
% { \vspace{-0.3cm}
%   \separator
%   \vspace{-0.25cm}
% }
% {
%   \vspace{-0.5cm}
%   \separator
%   \vspace{-0.15cm}
% }

% \def\A{\mathcal A}
% \def\oA{\overline{\mathcal A}}
% \def\S{\mathcal S}
% \def\D{\mathcal D}
% \def\eff{{\rm Eff}}
% \def\bD{\bm{D}}
% \def\cU{{\cal U}}
% \def\bbs{{\mathbb{s}}}
% \def\bbS{{\mathbb{S} }}
% \def\cM{{\cal M}}
% \def\bV{{\bm{V}}}
% \def\cH{{\cal H}}
% \def\ch{{\cal h}}
% \def\cR{{\cal R}}
% \def\cV{{\cal V}}
% \def\cA{{\cal A}}
% \def\cX{{\cal X}}
% \def\cN{{\cal N}}
% \def\cJ{{\cal J}}
% \def\cK{{\cal K}}
% \def\cL{{\cal L}}
% \def\cI{{\cal I}}
% \def\cY{{\cal Y}}
% \def\cZ{{\cal Z}}
% \def\cC{{\cal C}}
% \def\cR{{\cal R}}
% \def\id{{\rm Id}}
% \def\st{{\rm st}}
% \def\cF{{\cal F}}
% \def\bz{{\bm z}}
% \def\cG{{\cal G}}
% \def\N{\mathbb{N}}
% \def\bbh{\mathbb{h}}
% \def\bbH{\mathbb{H}}
% \def\bbi{\mathbb{i}}
% \def\bbI{\mathbb{I}}
% \def\R{\mathbb{R}}
% \def\bbR{\mathbb{R}}
% \def\bbr{\mathbb{r}}
% \def\cB{{\cal B}}
% \def\cP{{\cal P}}
% \def\cS{{\cal S}}
% \def\bW{{\bm W}}
% \def\bc{{\bm c}}

% %\def\and{\quad\mbox{and}\quad}
% \def\ind{{\bf 1}}


% \def\bmg{{\bm{\gamma}}}
% \def\bmr{{\bm{\rho}}}
% \def\bmq{{\bm{q}}}
% \def\bmt{{\bm{\tau}}}
% \def\bmn{{\bm{n}}}
% \def\bmcapn{{\bm{N}}}
% \def\bmrho{{\bm{\rho}}}

% \def\igam{\underline{\gamma}(\lambda)}
% \def\sgam{\overline{\gamma}(\lambda)}
% \def\ovt{\overline{\theta}}
% \def\ovT{\overline{\Theta}}
% \def\PP{{\mathrm P}}
% \def\EE{{\mathrm E}}
% \def\iskip{{\vskip -0.4cm}}
% \def\siskip{{\vskip -0.2cm}}

% \def\bp{\noindent{\it Proof.}\ }
% \def\ep{\hfill $\Box$}


\begin{document}

%itemshape
\setbeamertemplate{itemize item}{\scriptsize\raise1.25pt\hbox{\donotcoloroutermaths$\bullet$}}
\setbeamertemplate{itemize subitem}{\tiny\raise1.5pt\hbox{\donotcoloroutermaths$\circ$}}
\setbeamertemplate{itemize subsubitem}{\tiny\raise1.5pt\hbox{\donotcoloroutermaths$\blacktriangleright$}}
%default value for spacing
\plitemsep 0.1in
\pltopsep 0.03in
\setlength{\parskip}{0.15in}
%\setlength{\parindent}{-0.5in}
\setlength{\abovedisplayskip}{0.07in}
\setlength{\mathindent}{0cm}
\setbeamertemplate{frametitle continuation}{[\insertcontinuationcount]}

\setlength{\leftmargini}{0.5cm}
\setlength{\leftmarginii}{0.5cm}

\setlength{\fboxrule}{0.05pt}
\setlength{\fboxsep}{5pt}

\begin{frame}
  \titlepage
\end{frame}


% % Uncomment these lines for an automatically generated outline.
% \begin{frame}{Outline}
% % \tableofcontents
% \plitemsep 0.1in
% \bci
% \item 

% \item 
% \eci
% \end{frame}

% START START START START START START START START START START START START START

%%%%%%%%%%%%%%%%%%%%%%%%%%%%%%%%%%%%%%%%%%%%%%%%%%%%%%
\begin{frame}{Roadmap}

\plitemsep 0.1in

\bci 
\item Most remarkable two results in probability theory history

\bigskip

\item Weak Law of Large Numbers: Result and Meaning
\item Central Limit Theorem: Result and Meaning
\item Weak Law of Large Numbers: Proof

- Inequalities: Markov and Chebyshev

\item Central Limit Theorem: Proof

- Moment Generating Function (MGF)

\item Strong Law of Large Numbers
\eci 

\end{frame}

%%%%%%%%%%%%%%%%%%%%%%%%%%%%%%%%%%%%%%%%%%%%%%%%%%%%%%
\begin{frame}{Roadmap}

\plitemsep 0.1in

\bci 
\item Most remarkable two results in probability theory history

\bigskip

\item \redf{Weak Law of Large Numbers: Result and Meaning}
\item Central Limit Theorem: Result and Meaning
\item Weak Law of Large Numbers: Proof

- Inequalities: Markov and Chebyshev

\item Central Limit Theorem: Proof

- Moment Generating Function (MGF)

\item Strong Law of Large Numbers
\eci 

\end{frame}

%%%%%%%%%%%%%%%%%%%%%%%%%%%%%%%%%%%%%%%%%%%%%%%%%%%%%%
\begin{frame}{Our interest: Sum of Random Variables}

\plitemsep 0.1in

\bci 

\item<1-> \bluef{Example 1.} $n$ students who decides their presence, depending on their feeling. Each student is happy or sad at random. How many students will show their presence?

\item<2-> \bluef{Example 2.} I am hearing some sound. There are $n$ noisy sources from outside. 

\bigskip

\item<3-> $X_1, X_2, \ldots, X_n$: i.i.d (independent and identically distributed) random variables

\item<4-> $\expect{X_i} = \mu,$ $\var{X_i} = \sigma^2$ 

\item<5-> Our interest is to understand how the following sum behaves:
$$
\redf{S_n = X_1 + X_2 + \ldots + X_n}
$$

% \bci
% \item Challenging if we intend to approach directly. For $Z=X+Y,$ finding the distribution requires the complex \redf{convolution}. 
% \aleq{
% \pz = \cprob{X+Y = z} = \sum_{x}\px p_Y(z-x)
% }
% \eci
\eci 

\end{frame}

%%%%%%%%%%%%%%%%%%%%%%%%%%%%%%%%%%%%%%%%%%%%%%%%%%%%%%
\begin{frame}{Our First Strategy}

$$
S_n = X_1 + X_2 + \ldots + X_n
$$
\vspace{-0.8cm}
\plitemsep 0.1in
\bci 
\item<2-> Challenging if we intend to approach directly. Even just for $Z=X+Y,$ finding the distribution, for example, requires the complex \redf{convolution}. 
\aleq{
\pz = \cprob{X+Y = z} = \sum_{x}\px p_Y(z-x)
}

\item<3-> Take a certain \bluef{scaling} with respect to $n$ that corresponds to a \bluef{new glass}, and investigate the system for large $n$ 
% (i.e., $n \rightarrow \infty$ mathematically).

\item<4-> First, consider the sample mean, and try to understand how it behaves:
$$
M_n = \frac{X_1 + X_2+ \ldots X_n}{n}
$$

\eci 

\end{frame}

%%%%%%%%%%%%%%%%%%%%%%%%%%%%%%%%%%%%%%%%%%%%%%%%%%%%%%
\begin{frame}{Sample Mean}

$$
M_n = \frac{S_n}{n} = \frac{X_1 + X_2+ \ldots X_n}{n}
$$

\plitemsep 0.1in

\bci 
\item<2-> \redf{Example.} $n$ coin tossing. $X_i=1$ if head, and 0 otherwise. $S_n$: total number of heads.

\item<3-> $\cexpect{M_n} = \mu,$ $\cvar{M_n} = \sigma^2/n$

\item<4-> For large $n$, the variance decays. We expect that, for large $n,$ $M_n$ looses its randomness and concentrates around $\mu.$

\item<5-> Why important? If we take the scaling of $S_n$ by $1/n,$ it behaves like a deterministic number. This significantly simplifies how we understand the world.

\item<6-> We call this \redf{law of large numbers}.
\eci 

\end{frame}

%%%%%%%%%%%%%%%%%%%%%%%%%%%%%%%%%%%%%%%%%%%%%%%%%%%%%%
\begin{frame}{Let's Establish Mathematically}

$$
M_n = \frac{S_n}{n} = \frac{X_1 + X_2+ \ldots X_n}{n}
$$

\plitemsep 0.1in
\bci 
\item<2-> What about this? What's wrong?
$$
M_n \xrightarrow{n \rightarrow \infty} \mu
$$

\item<3-> Ordinary convergence for the sequence of real numbers: $a_n \rightarrow a$

\onslide<4->{- For every $\epsilon >0,$ there exists $n_0,$ such that for every $n \geq n_0,$ $|a_n-a| \le \epsilon.$}


\item<5-> $M_n$ is a random variable, which is a function from $\Omega$ to $\real.$

\item<6-> Need to mathematically build up the concept of convergence for the sequence of random variables.

\eci 

\end{frame}

%%%%%%%%%%%%%%%%%%%%%%%%%%%%%%%%%%%%%%%%%%%%%%%%%%%%%%
\begin{frame}{Convergence in Probability}

\plitemsep 0.1in

\bci 
\item<1-> Consider the sequence of rvs $(Y_n)_{n=1, 2, \ldots},$ and I want to say they ``converge'' to a number $a.$

\item<2-> Play the game with my friend Lin. 
\bci
\item<3-> Lin, give me any $\epsilon>0.$ 
\item<4-> OK. Then, let me consider the event $\{|Y_n -a| \geq \epsilon \},$ and compute its probability $a_n = \cprob{|Y_n -a| \geq \epsilon}.$
\item<5-> Now, $a_n$ is just the real number, and I will show that $a_n \rightarrow a$ as $n \rightarrow \infty.$

\onslide<6->{
\myblock{Convergence in probability}
{
For any $\epsilon >0,$ $\cbprob{|Y_n -a| \geq \epsilon} \xrightarrow{n \rightarrow \infty} 0.$
}}
\eci
\eci 
\end{frame}

%%%%%%%%%%%%%%%%%%%%%%%%%%%%%%%%%%%%%%%%%%%%%%%%%%%%%%
\begin{frame}{Weak Law of Large Numbers}
$$
M_n = \frac{S_n}{n} = \frac{X_1 + X_2+ \ldots X_n}{n}
$$
\onslide<2->{
\myblock{Weak law of large numbers}
{
$M_n$ converges to $\mu$ in probability.
}}

 \plitemsep 0.1in
 \bci 
 \item<3-> Why "Weak"? There exists a stronger stronger version, which we call ``strong" law of large numbers.

\item<4-> Proof requires some knowledge about useful inequalities, which we cover later. 
 \eci 
\end{frame}

%%%%%%%%%%%%%%%%%%%%%%%%%%%%%%%%%%%%%%%%%%%%%%%%%%%%%%
\begin{frame}{Roadmap}

\plitemsep 0.1in

\bci 
\item Most remarkable two results in probability theory history

\bigskip

\item \bluef{Weak Law of Large Numbers: Result and Meaning}
\item \redf{Central Limit Theorem: Result and Meaning}
\item Weak Law of Large Numbers: Proof

- Inequalities: Markov and Chebyshev

\item Central Limit Theorem: Proof

- Moment Generating Function (MGF)

\item Strong Law of Large Numbers
\eci 

\end{frame}


%%%%%%%%%%%%%%%%%%%%%%%%%%%%%%%%%%%%%%%%%%%%%%%%%%%%%%
\begin{frame}{Central Limit Theorem: Start with Scaling (1)}
% $$
% M_n = \frac{S_n}{n} = \frac{X_1 + X_2+ \ldots X_n}{n}
% $$
% \myblock{Weak law of large numbers}
% {
% $M_n$ converges to $\mu$ in probability.
% }

\plitemsep 0.1in
 \bci 
 \item<1-> Loosely speaking, WLLG says:
\aleq{
(M_n - \mu) \xrightarrow{n \rightarrow \infty} 0
}

\item<2-> However, we don't know \redf{how} $M_n -\mu$ converges to 0. For example, what's the speed of convergence? 

\item<3-> \redf{Question.} What should be ``something"? Something should what blows up. 
\aleq{
\onslide<3->{\redblk{\text{(something)}} & \times (M_n - \mu) \xrightarrow{n \rightarrow \infty} \text{meaningful thing}}\cr
\onslide<4->{\redblk{$n^\alpha$} &\times (M_n - \mu) \xrightarrow{n \rightarrow \infty} \text{meaningful thing}}
}
\item<5-> What's $\alpha$ for our magic? 

\item<6-> The answer is $\redf{\mfrac{1}{2}}$

\eci 
\end{frame}

%%%%%%%%%%%%%%%%%%%%%%%%%%%%%%%%%%%%%%%%%%%%%%%%%%%%%%
\begin{frame}{Central Limit Theorem: Start with Scaling (2)}

\plitemsep 0.1in
\bci 
\item Reshaping the equation:
\aleq{
\sqrt{n} &\times (M_n - \mu) = \onslide<2->{\sqrt{n}\lf(\frac{S_n - n\mu}{n}\ri) = \frac{S_n - n\mu}{\sqrt{n}}.} \quad \onslide<3->{\text{Let} \ Z_n = \frac{S_n - n\mu}{\bluef{\sigma} \sqrt{n}}.}
}

\item<4-> $\expect{Z_n} = 0$ and $\cvar{Z_n} = 1.$ 

- $Z_n$ is well-centered with a constant variance irrespective of $n.$

\item<5-> We expect that $Z_n$ converges to something meaningful, but what?

\item<6-> Some deterministic number just like WLLG? 

\item<7-> Interestingly, it converges to some \redf{random variable $Z$} that we know very well. 
\eci 
\end{frame}

%%%%%%%%%%%%%%%%%%%%%%%%%%%%%%%%%%%%%%%%%%%%%%%%%%%%%%
\begin{frame}{Central Limit Theorem: Formalism}

\plitemsep 0.05in
\bci 
\item<2-> $Z_n \xrightarrow{n \rightarrow \infty} Z,$ where $Z \sim N(0,1).$

\item<3-> Wait! What kind of convergence? Convergence in probability as in WLLN? No.

\item<4-> Convergence \redf{in distribution} (another type of convergence of rvs)
\onslide<5->{
\myblock{Central Limit Theorem}
{
For every $z,$ 
$$
\cprob{Z_n \le z} \xrightarrow{n \rightarrow \infty} \cprob{Z \le z},
$$
where $Z \sim N(0,1).$
}}

\item<6-> \redf{Meaning from scaling perspective.} 
\bci
\item<7-> LLN: Scaling $S_n$ by $1/n,$ you go to a deterministic world.
\item<8-> CLT: Scaling $S_n$ by $1/\sqrt{n}$, you still stay at the random world, but not an arbitrary random world. That's the normal random world, not depending on the distribution of each $X_i.$ Very interesting! 
\eci

\eci 
\end{frame}

%%%%%%%%%%%%%%%%%%%%%%%%%%%%%%%%%%%%%%%%%%%%%%%%%%%%%%
\begin{frame}{Practical Use of CLT}

$Z_n = \frac{S_n - n\mu}{\sigma \sqrt{n}}$

$\cprob{Z_n \le z} \xrightarrow{n \rightarrow \infty} \cprob{Z \le z}, \ Z \sim N(0,1)$

\plitemsep 0.1in
\bci 
\item<2-> Can approximate $Z_n$ with a standard normal rv
\item<4-> Can approximate $S_n$ with a normal rv $\sim  (n\mu, n\sigma^2)$

- $S_n = n \mu + Z_n\sigma\sqrt{n}$
\item<5-> How large should $n$ be?
\bci
\item<6-> A moderate $n$ (20 or 30) usually works, which the power of CLT.

\item<7-> If $X_i$ resembles a normal rv more, smaller $n$ works: symmetry and unimodality\footnotemark
\footnotetext{Only unique mode. A single maximum or minimum.}
\eci
\eci 


\end{frame}


%%%%%%%%%%%%%%%%%%%%%%%%%%%%%%%%%%%%%%%%%%%%%%%%%%%%%%
\begin{frame}{CLT: Examples of $n$}

\centering
\mypic{0.92}{L7_CTL_ex1.png}


\mypic{0.9}{L7_CTL_ex2.png}
\end{frame}



%%%%%%%%%%%%%%%%%%%%%%%%%%%%%%%%%%%%%%%%%%%%%%%%%%%%%%
\begin{frame}{Roadmap}

\plitemsep 0.1in

\bci 
\item Most remarkable two results in probability theory history

\bigskip

\item \bluef{Weak Law of Large Numbers: Result and Meaning}
\item \bluef{Central Limit Theorem: Result and Meaning}
\item \redf{Weak Law of Large Numbers: Proof

- Inequalities: Markov and Chebyshev}

\item Central Limit Theorem: Proof

- Moment Generating Function (MGF)

\item Strong Law of Large Numbers
\eci 

\end{frame}


%%%%%%%%%%%%%%%%%%%%%%%%%%%%%%%%%%%%%%%%%%%%%%%%%%%%%%
\begin{frame}{Markov Inequality}

\plitemsep 0.1in
\bci 
\item<1-> \redf{(Q)} Knowing $\cexpect{X},$ can we say something about the distribution of $X$?

\item<2-> Intuition: small $\cexpect{X}$ $\imp$ small $\cprob{X \ge a}$

\onslide<3->{
\myblock{Markov Inequality}
{
If $X \geq 0$ and $a >0,$ then $\cprob{X \geq a} \le \frac{\cexpect{X}}{a}.$
}}
\mytwocols{0.4}
{
\onslide<4->{
\redf{Proof.} For any $a>0,$ define $Y_a$ as:
\aleq{
Y_a &\eqdef \begin{cases}
0, & \text{if} \ X < a,\cr
a,& \text{if} \ X \ge a
\end{cases}
}}
\onslide<5->{Then, using non-negativity of $X,$ $Y_a \le X,$ which leads to $\expect{Y_a} \leq \expect{X}.$ }
}
{
\onslide<6->{Note that we have:
\aleq{
\expect{Y_a} = a \cprob{Y_a = a} = a \cprob{X \ge a}.
}}
\onslide<7->{Thus, $a \cdot \cprob{X \ge a} \leq \expect{X}.$} \qed

% \medskip
% - \redf{(Q)} Why do we need $X >0$?
}

\eci 
\end{frame}

%%%%%%%%%%%%%%%%%%%%%%%%%%%%%%%%%%%%%%%%%%%%%%%%%%%%%%
\begin{frame}{Chebyshev Inequality}

\plitemsep 0.1in
\bci 
\item<2-> \redf{(Q)} Knowing both $\cexpect{X}$ and $\cvar{X},$ can we say something about the distribution of $X$?

\item<3-> Intuition: small $\cvar{X}$ $\imp$ $X$ is unlikely to be too far away from its mean. 

\item<4-> $\cexpect{X} = \mu,$ $\cvar{X} = \sigma^2.$
\onslide<5->{\myblock{Chebyshev Inequality}
{
$$
\cbprob{|X-\mu| \geq c} \leq \frac{\sigma^2}{c^2}
$$
}}

\item<6-> \redf{Proof.} 
\aleq{
\onslide<6->{\cbprob{|X-\mu| \geq c} = \cbprob{(X-\mu)^2 \geq c^2} \leq} \onslide<7->{\frac{\bexpect{(X-\mu)^2}}{c^2} = \frac{\cvar{X}}{c^2}}
}


\eci 
\end{frame}

%%%%%%%%%%%%%%%%%%%%%%%%%%%%%%%%%%%%%%%%%%%%%%%%%%%%%%
\begin{frame}{Example}

\onslide<1->{- $X \sim \exp(1).$ Then, $\expect{X} = 1$ and $\var{X} =1.$} 

\onslide<2->{- $\cexpect{X \ge a} = e^{-a}$}

\bigskip
\mytwocols{0.3}
{
\plitemsep 0.1in
\bci 
\item<3-> \redf{Markov inequality}
\aleq{
\cprob{X \ge a} &\le \frac{\expect{X}}{a} = \frac{1}{a}
}
\eci 
}
{
\plitemsep 0.1in
\bci 
\item<4-> \redf{Chebyshev inequality}
\aleq{
\cprob{X \ge a} &= \cprob{X-1 \ge a-1} \cr
&\le \cprob{|X-1| \ge a-1} \le \frac{1}{(a-1)^2}
}
\eci 
}

\onslide<5->{- For reasonably large $a,$ CI provides much better bound. }

\onslide<6->{- knowing the variance helps}

\onslide<7->{- Both bounds are the ones that bound the \bluef{probability of rare events.}} 

\end{frame}

%%%%%%%%%%%%%%%%%%%%%%%%%%%%%%%%%%%%%%%%%%%%%%%%%%%%%%
\begin{frame}{Back to WLLN}
$$
M_n = \frac{S_n}{n} = \frac{X_1 + X_2+ \ldots X_n}{n}
$$
\myblock{Weak law of large numbers}
{
$M_n$ converges to $\mu$ in probability.
}


\redf{Proof.}

\onslide<2->{\aleq{
\cbprob{|M_n - \mu| \ge \epsilon} \le \frac{\cvar{M_n}}{\epsilon^2} = \frac{\sigma^2}{n \epsilon^2} \xrightarrow{n \rightarrow \infty} 0
}}

\end{frame}

%%%%%%%%%%%%%%%%%%%%%%%%%%%%%%%%%%%%%%%%%%%%%%%%%%%%%%
\begin{frame}{Roadmap}

\plitemsep 0.1in

\bci 
\item Most remarkable two results in probability theory history

\bigskip

\item \bluef{Weak Law of Large Numbers: Result and Meaning}
\item \bluef{Central Limit Theorem: Result and Meaning}
\item \bluef{Weak Law of Large Numbers: Proof

- Inequalities: Markov and Chebyshev
}

\item \redf{Central Limit Theorem: Proof

- Moment Generating Function (MGF)}

\item Strong Law of Large Numbers
\eci 

\end{frame}


%%%%%%%%%%%%%%%%%%%%%%%%%%%%%%%%%%%%%%%%%%%%%%%%%%%%%%
\begin{frame}{Moment Generating Function}

\mytwocols{0.7}
{
\small
\plitemsep 0.1in
\bci 
\item<1-> For a rv $X,$ we introduce a kind of transform, called \redf{moment generating function (MGF).} 

\item<2-> A function of a scalar parameter $s$, defined by
$$
M_X(s) = \expect{e^{sX}}
$$
- If clear, we omit $X$ and use $M(s).$
\aleq{
\onslide<3->{
M(s) &= \sum_x e^{sx} \px \quad \text{(discrete)}\cr
M(s) &= \int e^{sx} \fx dx \quad \text{(continuous)}\cr
}}
\eci 
}
{
\small
\plitemsep 0.1in
\bci 
\item<4->[Ex1)] $X \sim \exp(\lambda),$ $\fx = \lambda \elambdax, x\ge 0$
\aleq{
M(s) &= \lambda \int_{0}^\infty e^{sx} \elambdax dx \cr
&= \lambda \frac{e^{(s-\lambda)x}}{s-\lambda} \Bigg |_0^\infty \quad (\text{if  } s <\lambda) \cr
&= \frac{\lambda}{\lambda - s}
}

\item<5->[Ex2)] $X \sim N(0,1)$ (homework problem)
\aleq{
M(s) = e^{s^2/2}
}

\eci 
}

\end{frame}

%%%%%%%%%%%%%%%%%%%%%%%%%%%%%%%%%%%%%%%%%%%%%%%%%%%%%%
\begin{frame}{Useful Properties of MGF}

\plitemsep 0.1in
\bce 
\item<1-> $M'(0) = \expect{X}$
\aleq{
\onslide<2->{\frac{d}{ds} M(s) &=\frac{d}{ds} \int_{-\infty}^\infty e^{sx} \fx dx = \int_{-\infty}^\infty \frac{d}{ds} e^{sx} \fx dx = \int_{-\infty}^\infty x e^{sx} \fx dx \cr
&= \frac{d}{ds} M(s) \Bigg |_{s=0} = \expect{X}}
}

\item<3-> Similarly, $M''(0) = \expect{X^2}$

\item<4-> $\frac{d^n}{ds^n} M(s) \Bigg |_{s=0} = \expect{X^n}$

\item<5-> MGF provides a convenient way of generating moments. That's why it is called moment generating function. 

\ece 


\end{frame}

%%%%%%%%%%%%%%%%%%%%%%%%%%%%%%%%%%%%%%%%%%%%%%%%%%%%%%
\begin{frame}{Inversion Property}

\myblock{Inversion Property}
{
The transform $M_X(s)$ associated with a random variable $X$ uniquely determines 
the CDF of $X,$ assuming that $M_X(s)$ is finite for all $s$ in some interval $[-a,a],$ where $a$ is a positive number.
}

\plitemsep 0.1in
\bci 
\item In easy words, we can recover the distribution if we know the MGF. 

\item Thus, each rv has its own MGF. 

\eci 


\end{frame}


%%%%%%%%%%%%%%%%%%%%%%%%%%%%%%%%%%%%%%%%%%%%%%%%%%%%%%
\begin{frame}{Back to CLT}

\plitemsep 0.05in
\bci 
\item<1->  Without loss of generality, assume $\cexpect{X_i}=0$ and $\cvar{X_i} = 1$

\item<2->  $Z_n = \frac{S_n}{\sqrt{n}} = \frac{X_1 + X_2 + \ldots X_n}{\sqrt{n}}$

\item<3->  Show: MGF of $Z_n$ converges to MFG of $N(0,1)$ (using inversion property)
\eci


\mytwocols{0.6}
{
\small
\redf{Proof.}
\aleq{
\onslide<4->{\bexpect{e^{s S_n/\sqrt{n}}} &= \bexpect{e^{s X_1/\sqrt{n}}}\times \cdots \times\bexpect{e^{s X_n/\sqrt{n}}}}\cr
&=\onslide<5->{\lf ( \bexpect{e^{s X_1/\sqrt{n}}} \ri )^n} = 
\onslide<6->{\lf( M_{X_1}\Bl(\frac{s}{\sqrt{n}}\Bl) \ri)^n}
}
\onslide<7->{- For simplicity, let $M(\cdot) = M_{X_1}(\cdot)$}

\onslide<8->{- \bluef{Facts:} $M(0) = 1$, $M'(0) = 0,$ $M''(0) = 1$}

\onslide<9->{- $\lf( M\Bl(\frac{s}{\sqrt{n}}\Bl) \ri)^n \rightarrow \text{what???}$}

\onslide<10->{- Taking log, $n \log M\Bl(\frac{s}{\sqrt{n}}\Bl)  \rightarrow \text{what???}$}
}
{
\onslide<11->{ For convenience, do the change of variable $y = \frac{1}{\sqrt{n}}.$ Then, we have
$$
\lim_{y\rightarrow 0}\frac{\log M (ys)}{y^2}
$$
}

\onslide<12->{- If we apply l'hopital's rule twice (please check), we get
\aleq{
\lim_{y\rightarrow 0}\frac{\log M (ys)}{y^2} = \frac{s^2}{2}
}}
\qed
}
\end{frame}


%%%%%%%%%%%%%%%%%%%%%%%%%%%%%%%%%%%%%%%%%%%%%%%%%%%%%%
\begin{frame}{Roadmap}

\plitemsep 0.1in

\bci 
\item Most remarkable two results in probability theory history

\bigskip

\item \bluef{Weak Law of Large Numbers: Result and Meaning}
\item \bluef{Central Limit Theorem: Result and Meaning}
\item \bluef{Weak Law of Large Numbers: Proof

- Inequalities: Markov and Chebyshev
}

\item \bluef{Central Limit Theorem: Proof

- Moment Generating Function (MGF)}

\item \redf{Strong Law of Large Numbers (Optional)}
\eci 

\end{frame}


%%%%%%%%%%%%%%%%%%%%%%%%%%%%%%%%%%%%%%%%%%%%%%%%%%%%%%
\begin{frame}{}
\vspace{2cm}
\LARGE Questions?

\end{frame}

%%%%%%%%%%%%%%%%%%%%%%%%%%%%%%%%%%%%%%%%%%%%%%%%%%%%%%
\begin{frame}{Review Questions}

\bce[1)]
\item What's the practical value of LLN and CLT?

\item Explain LLN and CLT from the scaling perspective.

\item Why are LLN and CLT great?

\item Why do we need different concepts of convergence for random variables?

\item Explain what is convergence in probability.

\item Explain what is convergence in distribution. 

\item Why is MGF useful?

\ece

\end{frame}

\end{document}
